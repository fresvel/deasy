\section{Introduction}
El cáncer de mama representa un grave problema de salud pública a nivel mundial y constituye una de las principales causas de muerte por cáncer en mujeres cuando no se diagnostica a tiempo\cite{r-29,r-38}. En 2020 se reportaron aproximadamente 2,3 millones de nuevos casos, con una tasa de mortalidad cercana al 30\% entre las pacientes afectadas \cite{r-22}. En este contexto, la detección precoz resulta fundamental, pues mejora significativamente las opciones terapéuticas disponibles y aumenta las probabilidades de supervivencia \cite{r-29}. Las modalidades de imagen como la mamografía, la ecografía y la resonancia magnética son herramientas esenciales para el cribado y diagnóstico temprano\cite{r-29}. En los últimos años, la inteligencia artificial, y en particular el aprendizaje profundo, ha revolucionado la interpretación de imágenes médicas\cite{r-20,r-22}. Los algoritmos, especialmente las redes neuronales convolucionales, poseen una capacidad destacada para reconocer patrones complejos, sutiles y difíciles de identificar en grandes volúmenes de datos, en ocasiones superando el desempeño de operadores humanos entrenados\cite{r-20}. Estos sistemas pueden analizar imágenes médicas con fines de clasificación, segmentación y detección de anomalías con alta precisión\cite{r-22}. El vínculo entre los avances de la inteligencia artificial y su aplicabilidad en el cáncer de mama resulta directo y prometedor, ya que actualmente se desarrollan sistemas capaces de analizar mamografías e imágenes histopatológicas para detectar y clasificar masas tumorales de manera rápida y precisa\cite{r-22}. Se han propuesto modelos de aprendizaje profundo con niveles de exactitud cercanos al 95\% en la clasificación de imágenes histológicas de cáncer de mama y en la detección de metástasis en ganglios linfáticos\cite{r-02,r-38}. La inteligencia artificial, además, permite automatizar procesos, reducir la dependencia del examen manual y disminuir las tasas de diagnóstico erróneo, consolidándose como una herramienta clave para mejorar la detección temprana y, en consecuencia, los resultados clínicos de las pacientes\cite{r-22,r-02}.

A pesar de estos avances, el diagnóstico clínico del cáncer de mama sigue enfrentando limitaciones significativas. La interpretación de imágenes, en especial las mamografías, es un proceso complejo y demandante que puede presentar tasas de error de hasta el 30\%\cite{r-38,r-22}. Entre las principales dificultades se encuentran la variabilidad en la interpretación debida a la subjetividad del evaluador, las características complejas e irregulares de las masas tumorales que dificultan su segmentación y clasificación, la baja especificidad de ciertas modalidades como la mamografía digital, que genera altos índices de falsos positivos y biopsias innecesarias, y la considerable carga de trabajo que recae sobre radiólogos y patólogos, lo que hace que el proceso sea más lento y susceptible de errores\cite{r-38,r-22}. Aunque múltiples estudios han reportado un alto rendimiento de los sistemas de inteligencia artificial, persisten importantes vacíos de evidencia respecto a su efectividad real en la práctica clínica frente a los métodos convencionales. La mayoría de los modelos desarrollados no llegan a implementarse por problemas de generalización, variabilidad de los datos o falta de herramientas adaptadas al entorno clínico\cite{r-30,r-22}. Además, la percepción de estos algoritmos como “cajas negras” genera desconfianza, pues los médicos necesitan comprender la lógica detrás de una clasificación para poder validarla y adoptarla\cite{r-38}. En consecuencia, resulta crucial evaluar si las métricas de rendimiento reportadas en la literatura, como la exactitud, la sensibilidad o el área bajo la curva, se traducen en una utilidad clínica tangible\cite{r-20}. Asimismo, es necesario establecer comparaciones rigurosas entre los sistemas de inteligencia artificial y las decisiones de expertos humanos, y garantizar una validación clínica sólida mediante marcos regulatorios y estándares de calidad que aún no se encuentran plenamente consolidados, lo que limita la traslación de estas tecnologías a la práctica médica diaria \cite{r-20,r-22,r-38}.

El objetivo principal de este estudio es evaluar la eficacia diagnóstica de los sistemas de inteligencia artificial aplicados al análisis de imágenes médicas para la detección del cáncer de mama mediante una revisión sistemática con metaanálisis. Como objetivos secundarios se propone comparar el desempeño de los sistemas de inteligencia artificial frente a los métodos convencionales y a los radiólogos con el fin de determinar si aportan mejoras significativas en relación con la práctica estándar, sintetizar las métricas de rendimiento reportadas en los estudios incluidos para obtener una medida global de su eficacia, analizar la heterogeneidad de los resultados y los posibles sesgos de la literatura considerando factores como el tipo de modelo, la modalidad de imagen, las características de los conjuntos de datos y el diseño metodológico, así como determinar la aplicabilidad clínica de los hallazgos en diferentes contextos, incluyendo hospitales, centros de cribado y entornos con recursos limitados, evaluando los desafíos que supone su implementación en términos de infraestructura e integración en los flujos de trabajo clínicos.



%The introduction should briefly place the study in a broad context and highlight why it is important. It should define the purpose of the work and its significance. The current state of the research field should be reviewed carefully and key publications cited. Please highlight controversial and diverging hypotheses when necessary. Finally, briefly mention the main aim of the work and highlight the principal conclusions. As far as possible, please keep the introduction comprehensible to scientists outside your particular field of research. Citing a journal paper \citep{ref-journal}.  Now citing a book reference \citep{ref-book1,ref-book2} or other reference types \citep{ref-unpublish,ref-url}. Please use the command \citep{ref-proceeding,ref-thesis} for the following MDPI journals, which use author--date citation: Administrative Sciences, Arts, Behavioral Sciences, Businesses, Econometrics, Economies, Education Sciences, European Journal of Investigation in Health, Psychology and Education, Games, Genealogy, Histories, Humanities, Humans, IJFS, Journal of Intelligence, Journalism and Media, JRFM, Languages, Laws, Literature, Psychology International, Publications, Religions, Risks, Social Sciences, Tourism and Hospitality, Youth. 

%%%%%%%%%%%%%%%%%%%%%%%%%%%%%%%%%%%%%%%%%%
