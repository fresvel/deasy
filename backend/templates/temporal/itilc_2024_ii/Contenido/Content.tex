\small
\begin{tabularx}{\textwidth}{|p{2.5cm}|p{2.5cm}|X|X|X|X|}
\hline
\multicolumn{6}{|X|}{\textbf{Nivel: 01 }}\\\hline\textbf{Materia} & \textbf{Docente} & \textbf{Estudiantes} & \textbf{Aprobados} & \textbf{Promedio} & \textbf{\%Supera el Promedio} \\ \hline
Matemáticas I & BUSTOS VERA RHAY PABLO & 11 & 8 & 35.91 & 72.73 \%\\ \hline
TICs & VELASTEGUI IZURIETA HOMERO JAVIER & 11 & 8 & 35.64 & 72.73 \%\\ \hline
Programación I & POSLIGUA FLORES KLEBER ROLANDO & 11 & 8 & 28.09 & 72.73 \%\\ \hline
Sistemas Operativos & CARVAJAL CARVAJAL JOSE LUIS & 11 & 8 & 30 & 54.55 \%\\ \hline
Álgebra & HIDALGO SOLORZANO LUIS ENRIQUE & 11 & 8 & 29.36 & 72.73 \%\\ \hline
Habilidades Profesionales & SAYAGO HEREDIA JAIME PAUL & 11 & 8 & 34.27 & 72.73 \%\\ \hline
Tutorías de Acompañamiento & VELASTEGUI IZURIETA HOMERO JAVIER & 11 & 0 & 0 & 0.00 \%\\ \hline
\end{tabularx}

\vspace{1cm}
\section{Análisis de Rendimiento}
El análisis del rendimiento académico de los estudiantes del primer nivel de TICs muestra una tendencia general positiva, con un porcentaje de aprobados elevado en la mayoría de las materias. En general, el promedio del curso es de alrededor de 34-35 puntos, lo que indica una buena calidad en el rendimiento académico.

En cuanto a las materias específicas, se observa que las materias de Matemáticas I, Álgebra y Habilidades Profesionales tienen el mismo número de estudiantes aprobados, es decir, 8 de 11. Las materias de TICs, Programación I y Sistemas Operativos también tienen un número elevado de aprobados, 8 de 11 y 8 de 11 respectivamente.

En cuanto al porcentaje que supera el promedio, se observa que las materias de Matemáticas I, Álgebra, Habilidades Profesionales, TICs, Programación I y Sistemas Operativos y Jesucristo tienen un porcentaje elevado, entre 72.73\% y 88.89\%, lo que indica una buena calidad en el rendimiento académico.

En resumen, el análisis del rendimiento académico del primer nivel de TICs muestra una tendencia general positiva, con un porcentaje de aprobados elevado en la mayoría de las materias y un promedio del curso aceptable. Se destaca la buena calidad en las materias de Matemáticas I, Álgebra, Habilidades Profesionales, TICs, Programación I, Sistemas Operativos y Jesucristo y la Persona Hoy, y se recomienda seguir monitoreando el rendimiento de las materias que tienen un porcentaje más bajo de aprobados.\\
\vspace{1cm}

