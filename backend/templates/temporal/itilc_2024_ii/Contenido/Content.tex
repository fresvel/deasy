\small
\begin{tabularx}{\textwidth}{|p{2.5cm}|p{2.5cm}|X|X|X|X|}
\hline
\multicolumn{6}{|X|}{\textbf{Nivel: 02 }}\\\hline\textbf{Materia} & \textbf{Docente} & \textbf{Estudiantes} & \textbf{Aprobados} & \textbf{Promedio} & \textbf{\%Supera el Promedio} \\ \hline
Química General & PEÑA ROSAS GLORIA DEL VALLE & 12 & 0 & 0 & 0.00 \%\\ \hline
Química Analítica & PEÑA ROSAS GLORIA DEL VALLE & 16 & 13 & 32.25 & 56.25 \%\\ \hline
Química Analítica & PEÑA ROSAS GLORIA DEL VALLE & 9 & 0 & 0 & 0.00 \%\\ \hline
Química Analítica & PEÑA ROSAS GLORIA DEL VALLE & 7 & 0 & 0 & 0.00 \%\\ \hline
Quimica Organica & ACOSTA GANAN MICHAEL ANDRES & 19 & 13 & 30.26 & 47.37 \%\\ \hline
Quimica Organica & ACOSTA GANAN MICHAEL ANDRES & 11 & 0 & 0 & 0.00 \%\\ \hline
Quimica Organica & ACOSTA GANAN MICHAEL ANDRES & 8 & 0 & 0 & 0.00 \%\\ \hline
Bioquimica & ACOSTA GANAN MICHAEL ANDRES & 16 & 8 & 30.5 & 43.75 \%\\ \hline
Bioquimica & ACOSTA GANAN MICHAEL ANDRES & 10 & 0 & 0 & 0.00 \%\\ \hline
Microbiología & CASTRO DEMERA DICKE ALEJANDRO & 19 & 8 & 24.47 & 52.63 \%\\ \hline
Microbiología & CASTRO DEMERA DICKE ALEJANDRO & 13 & 0 & 0 & 0.00 \%\\ \hline
Microbiología & CASTRO DEMERA DICKE ALEJANDRO & 6 & 0 & 0 & 0.00 \%\\ \hline
Tutorías de Acompañamiento & PEÑA ROSAS GLORIA DEL VALLE & 18 & 0 & 0 & 0.00 \%\\ \hline
Fundamentos de la Investigació & SUAREZ LEZCANO JOSE  & 12 & 9 & 39.33 & 75.00 \%\\ \hline
\end{tabularx}

\vspace{1cm}
\section{Análisis de Rendimiento}
El análisis del rendimiento del curso de primer nivel de TICs, revela una variedad de patrones y tendencias que merecen ser destacados.

En cuanto a la materia de Química General, no hay ningún estudiante que haya aprobado el curso, lo que sugiere un rendimiento pobre. Por otro lado, en la materia de Química Analítica, el resultado es mixto: hay dos secciones con un número considerable de aprobados, mientras que otras dos secciones no tienen nadie que haya aprobado el curso. En general, el promedio del curso de Química Analítica es de 32.25, lo que es aceptable.

En el caso de la materia de Quimica Organica, hay un patrón claro de tendencia: hubo una sección con un buen número de aprobados, mientras que las otras dos secciones no tuvieron nadie que aprobara el curso. El promedio del curso de Quimica Organica es de 30.26, lo que sugiere un rendimiento mediocre.

La materia de Bioquimica también presenta una tendencia mixta: hay una sección con un número considerable de aprobados, mientras que la otra sección no tuvo nadie que aprobara el curso. El promedio del curso de Bioquimica es de 30.5, que es similar al de Quimica Organica.

La materia de Microbiología también presenta una tendencia mixta similar a la de Bioquimica y Quimica Organica. Finalmente, la materia de Tutorías de Acompañamiento no tuvo nadie que aprobara el curso.

En cuanto al promedio general, el resultado más alto se obtuvo en la materia de Fundamentos de la Investigación, con un promedio del 39.33 y un porcentaje de aprobados del 75\%. Esto sugiere que la materia tiene un nivel de dificultad relativamente alto, pero los estudiantes que aprobaron están logrando un buen rendimiento.

En general, los resultados sugieren que hay algunas materias en las que los estudiantes están teniendo dificultades, como Química General y las materias de microbiología, bioquimica y quimica organica que no tienen aprobados en algunas secciones. Por otro lado, hay materias como Fundamentos de la Investigación que están resultando más atractivas para los estudiantes y en las que se está obteniendo un buen rendimiento. Estos resultados pueden ser útiles para los docentes y administradores para identificar áreas en las que necesitan invertir más recursos y apoyo para mejorar el rendimiento de los estudiantes.\\
\vspace{1cm}\\\small
\begin{tabularx}{\textwidth}{|p{2.5cm}|p{2.5cm}|X|X|X|X|}
\hline
\multicolumn{6}{|X|}{\textbf{Nivel: 01 }}\\\hline\textbf{Materia} & \textbf{Docente} & \textbf{Estudiantes} & \textbf{Aprobados} & \textbf{Promedio} & \textbf{\%Supera el Promedio} \\ \hline
Química General & PEÑA ROSAS GLORIA DEL VALLE & 26 & 23 & 33.42 & 57.69 \%\\ \hline
Química General & PEÑA ROSAS GLORIA DEL VALLE & 14 & 0 & 0 & 0.00 \%\\ \hline
Citología e Histología & RUEDA CASTILLO YAJAIRA MARILIN & 26 & 22 & 33.88 & 50.00 \%\\ \hline
Citología e Histología & RUEDA CASTILLO YAJAIRA MARILIN & 15 & 0 & 0 & 0.00 \%\\ \hline
Citología e Histología & RUEDA CASTILLO YAJAIRA MARILIN & 11 & 0 & 0 & 0.00 \%\\ \hline
Anatomía y Fisiología & ESTUPIÑAN SANCHEZ ALFREDO SANDINO & 25 & 24 & 40.32 & 60.00 \%\\ \hline
Anatomía y Fisiología & ESTUPIÑAN SANCHEZ ALFREDO SANDINO & 14 & 0 & 0 & 0.00 \%\\ \hline
Anatomía y Fisiología & ESTUPIÑAN SANCHEZ ALFREDO SANDINO & 11 & 0 & 0 & 0.00 \%\\ \hline
Técnicas Básicas de Laboratori & HIDALGO TAPIA COSME ENRIQUE & 14 & 14 & 34 & 42.86 \%\\ \hline
Técnicas Básicas de Laboratori & HIDALGO TAPIA COSME ENRIQUE & 11 & 10 & 32.91 & 72.73 \%\\ \hline
Tutorías de Acompañamiento & PEÑA ROSAS GLORIA DEL VALLE & 26 & 0 & 0 & 0.00 \%\\ \hline
Tecnologías de la Información  & RIVERA BONE CARLOS ALEJANDRO & 25 & 22 & 41.16 & 80.00 \%\\ \hline
Comunicación Oral y Escrita & CABALLERO MOREIRA JAIRON ADRIAN & 25 & 23 & 39.76 & 64.00 \%\\ \hline
\end{tabularx}

\vspace{1cm}
\section{Análisis de Rendimiento}
A continuación, se presenta un análisis del rendimiento académico de los estudiantes del primer nivel de TICs, en base a los datos presentados.

En general, se observa que la mayor parte de los estudiantes han logrado un buen rendimiento académico, con promedios superiores a 30. Solo en dos materias, Química General y Citología e Histología, se han reportado promedios muy bajos, en torno a 0.00. Estos resultados sugieren que estos estudiantes necesitan recibir apoyo adicional y ajustes en su estrategia de aprendizaje para mejorar su desempeño.

En cuanto a la materia Química General, se observa que la mayoría de los estudiantes (23 de 26) han logrado aprobar, lo que indica que el docente PEÑA ROSAS GLORIA DEL VALLE ha sido capaz de transmitir el conocimiento adecuadamente. Sin embargo, en la siguiente oportunidad la materia, se reportaron cero aprobados, lo que podría indicar que los estudiantes no estaban preparados o no comprendieron los conceptos básicos.

En Citología e Histología también se observa un buen número de aprobados (22 de 26), lo que sugiere que el docente RUEDA CASTILLO YAJAIRA MARILIN ha tenido éxito en transmitir el conocimiento, aunque con algunos estudiantes que no lograron aprobar. Sin embargo, en las dos últimas oportunidades, no hubo aprobados, lo que indica que los estudiantes necesitan recibir apoyo adicional y ajustes en su estrategia de aprendizaje.

En las demás materias, se observa un rendimiento académico satisfactorio, con promedios superiores a 30 y un buen número de aprobados. En Anatomía y Fisiología, se reporta un promedio de 40.32 y 60\% de aprobados, lo que sugiere que los estudiantes han logrado una buena comprensión de los conceptos. En Técnicas Básicas de Laboratorio, se observa un promedio de 34 y 42.86\% de aprobados, lo que indica que los estudiantes han logrado una buena mayoría aprobada, aunque algunos necesitan mejorar. En Tecnologías de la Información, se reporta un promedio de 41.16 y 80\% de aprobados, lo que sugiere que los estudiantes han logrado una buena comprensión de los conceptos y han superado el promedio. Por último, en Comunicación Oral y Escrita, se observa un promedio de 39.76 y 64\% de aprobados, lo que indica que los estudiantes han logrado una buena mayoría aprobada, aunque algunos necesitan mejorar.

En resumen, los resultados indican que la mayoría de los estudiantes del primer nivel de TICs han logrado un buen rendimiento académico, aunque hay algunas materias y students que necesitan recibir apoyo adicional y ajustes en su estrategia de aprendizaje.\\
\vspace{1cm}\\\small
\begin{tabularx}{\textwidth}{|p{2.5cm}|p{2.5cm}|X|X|X|X|}
\hline
\multicolumn{6}{|X|}{\textbf{Nivel: 05 }}\\\hline\textbf{Materia} & \textbf{Docente} & \textbf{Estudiantes} & \textbf{Aprobados} & \textbf{Promedio} & \textbf{\%Supera el Promedio} \\ \hline
Bioquimica Clínica II & HIDALGO TAPIA COSME ENRIQUE & 2 & 0 & 0 & 0.00 \%\\ \hline
Inmunología Clínica II & RUEDA CASTILLO YAJAIRA MARILIN & 3 & 0 & 0 & 0.00 \%\\ \hline
Bacteriología Clínica I & CASTRO DEMERA DICKE ALEJANDRO & 32 & 20 & 28.13 & 62.50 \%\\ \hline
Bacteriología Clínica I & CASTRO DEMERA DICKE ALEJANDRO & 14 & 0 & 0 & 0.00 \%\\ \hline
Bacteriología Clínica I & CASTRO DEMERA DICKE ALEJANDRO & 18 & 0 & 0 & 0.00 \%\\ \hline
Epidemiología y Realidad Nacio & ACOSTA GANAN MICHAEL ANDRES & 26 & 17 & 30.88 & 53.85 \%\\ \hline
Virología & ESTUPIÑAN SANCHEZ ALFREDO SANDINO & 22 & 22 & 41.77 & 63.64 \%\\ \hline
Bioquímica Clínica III & ZÚÑIGA SOSA EVELIN ALEXANDRA & 26 & 23 & 32.85 & 57.69 \%\\ \hline
Bioquímica Clínica III & ZÚÑIGA SOSA EVELIN ALEXANDRA & 13 & 0 & 0 & 0.00 \%\\ \hline
Bioquímica Clínica III & ZÚÑIGA SOSA EVELIN ALEXANDRA & 13 & 0 & 0 & 0.00 \%\\ \hline
Inmunohematología & CHILA GARCIA KAREN CAROLINA & 39 & 33 & 30.69 & 61.54 \%\\ \hline
Inmunohematología & CHILA GARCIA KAREN CAROLINA & 13 & 0 & 0 & 0.00 \%\\ \hline
Inmunohematología & CHILA GARCIA KAREN CAROLINA & 16 & 0 & 0 & 0.00 \%\\ \hline
Citología Clínica & ESTUPIÑAN SANCHEZ ALFREDO SANDINO & 25 & 25 & 43.16 & 48.00 \%\\ \hline
Citología Clínica & ESTUPIÑAN SANCHEZ ALFREDO SANDINO & 12 & 0 & 0 & 0.00 \%\\ \hline
Tutorías de Acompañamiento & ACOSTA GANAN MICHAEL ANDRES & 32 & 0 & 0 & 0.00 \%\\ \hline
Inmunohematología & CHILA GARCIA KAREN CAROLINA & 10 & 0 & 0 & 0.00 \%\\ \hline
\end{tabularx}

\vspace{1cm}
\section{Análisis de Rendimiento}
Se analizan los resultados del rendimiento académico de los estudiantes de primer nivel de TICs en diferentes materias, contabilizando un total de 215 Studentesmatriculados. De estos, 102 estudiantes aprobaron al menos un curso.

El curso con el mejor rendimiento se encuentra en "Virología" con un promedio de 41.77 y un 63.64\% de superación del promedio. En segundo lugar, se encuentra "Inmunohematología" con un promedio de 30.69 y un 61.54\% de superación del promedio. El tercer curso con mejor rendimiento es "Citología Clínica" con un promedio de 43.16 y un 48\% de superación del promedio.

Por otro lado, los cursos con peor rendimiento se encuentran en "Bioquímica Clínica II", "Inmunología Clínica II" y "Tutorías de Acompañamiento", todos ellos con un promedio de 0 y no aprobadados.

El curso de "Bacteriología Clínica I" tiene un promedio de 28.13 y un 62.50\% de superación del promedio, aunque hay tres secciones con distinto número de matriculados, siendo la sección más grande la que tiene un promedio de 32 y un 62.50\% de superación del promedio.

En general, se observa que la mayoría de los cursos tienen un promedio bajo y un porcentaje de superación del promedio bajo, lo que sugiere que los estudiantes necesitan apoyo adicional para mejorar su rendimiento.\\
\vspace{1cm}\\\small
\begin{tabularx}{\textwidth}{|p{2.5cm}|p{2.5cm}|X|X|X|X|}
\hline
\multicolumn{6}{|X|}{\textbf{Nivel: 03 }}\\\hline\textbf{Materia} & \textbf{Docente} & \textbf{Estudiantes} & \textbf{Aprobados} & \textbf{Promedio} & \textbf{\%Supera el Promedio} \\ \hline
Bioquimica & ACOSTA GANAN MICHAEL ANDRES & 6 & 0 & 0 & 0.00 \%\\ \hline
Parasitología & CHILA GARCIA KAREN CAROLINA & 19 & 12 & 28.53 & 63.16 \%\\ \hline
Parasitología & CHILA GARCIA KAREN CAROLINA & 9 & 0 & 0 & 0.00 \%\\ \hline
Parasitología & CHILA GARCIA KAREN CAROLINA & 10 & 0 & 0 & 0.00 \%\\ \hline
Análisis Instrumental & ACOSTA GANAN MICHAEL ANDRES & 14 & 8 & 27.79 & 57.14 \%\\ \hline
Análisis Instrumental & ACOSTA GANAN MICHAEL ANDRES & 9 & 8 & 32.78 & 44.44 \%\\ \hline
Inmunología Clínica I & CHILA GARCIA KAREN CAROLINA & 25 & 16 & 29.44 & 64.00 \%\\ \hline
Inmunología Clínica I & CHILA GARCIA KAREN CAROLINA & 10 & 0 & 0 & 0.00 \%\\ \hline
Inmunología Clínica I & CHILA GARCIA KAREN CAROLINA & 12 & 0 & 0 & 0.00 \%\\ \hline
Hematología I & AGREDA EGAS EYLEN AMANDA & 18 & 12 & 29.39 & 66.67 \%\\ \hline
Hematología I & AGREDA EGAS EYLEN AMANDA & 8 & 0 & 0 & 0.00 \%\\ \hline
Hematología I & AGREDA EGAS EYLEN AMANDA & 7 & 0 & 0 & 0.00 \%\\ \hline
Hematología I & AGREDA EGAS EYLEN AMANDA & 3 & 0 & 0 & 0.00 \%\\ \hline
Bioquímica Clínica I & HIDALGO TAPIA COSME ENRIQUE & 20 & 18 & 32.15 & 50.00 \%\\ \hline
Bioquímica Clínica I & HIDALGO TAPIA COSME ENRIQUE & 17 & 0 & 0 & 0.00 \%\\ \hline
Bioquímica Clínica I & HIDALGO TAPIA COSME ENRIQUE & 3 & 0 & 0 & 0.00 \%\\ \hline
Tutorías de Acompañamiento & CHILA GARCIA KAREN CAROLINA & 23 & 0 & 0 & 0.00 \%\\ \hline
Contextos e Interculturalidad & BAUTISTA MEJIA KATIUSKA MARIA & 21 & 18 & 38.1 & 61.90 \%\\ \hline
Jesucristo y la Persona de Hoy & QUINTERO ROSALES FRANCISCO JHONNY & 12 & 7 & 30.5 & 41.67 \%\\ \hline
\end{tabularx}

\vspace{1cm}
\section{Análisis de Rendimiento}
Los datos presentados permiten el análisis del rendimiento académico de los estudiantes de primer nivel de TICs en diferentes materias. En general, se observa que la mayoría de las materias tienen un número mayor de estudiantes aprobados que de estudiantes inscritos, lo que sugiere que la calidad de la educación es alta en la mayoría de los casos.

En específico, la materia de Parasitología de la profesora CHILA GARCIA KAREN CAROLINA, tiene un porcentaje de aprobados significativamente bajo, con un 28.53\% de los estudiantes que superaron el curso. Esto puede ser debido a que la materia es de dificultad alta y requiere una gran cantidad de conocimientos previos en biología y medicina. Por otro lado, la materia de Análisis Instrumental de la profesora ACOSTA GANAN MICHAEL ANDRES, tiene un porcentaje de aprobados moderado, con un 57.14\% de los estudiantes que superaron el curso. Es importante destacar que la materia de Análisis Instrumental es fundamental en la educación de los estudiantes de TICs, ya que se relaciona estrechamente con la utilización de tecnologías en el campo de la salud.

En la materia de Inmunología Clínica I de la profesora CHILA GARCIA KAREN CAROLINA, se observa un porcentaje de aprobados relativamente alto, con un 64\% de los estudiantes que superaron el curso. Esto sugiere que la calidad de la educación en esta materia es alta y que los estudiantes han desarrollado una comprensión adecuada de los conceptos relacionados con la inmunología.

En tercer lugar, la materia de Bioquímica Clínica I de la profesora HIDALGO TAPIA COSME ENRIQUE, tiene un porcentaje de aprobados moderado, con un 50\% de los estudiantes que superaron el curso. Sin embargo, es importante destacar que la materia de Bioquímica Clínica I es fundamental en la educación de los estudiantes de TICs, ya que se relaciona estrechamente con la Diagnosis y el tratamiento de enfermedades.

En resumen, los resultados de los datos presentados suggest que la mayoría de las materias tienen un porcentaje de aprobados relativamente alto, lo que sugiere que la calidad de la educación es alta en la mayoría de los casos. Sin embargo, hubo algunas materias que presentaron un porcentaje de aprobados más bajo, lo que sugiere que la educación debe ser revaluada y adaptada para mejorar los resultados.\\
\vspace{1cm}\\\small
\begin{tabularx}{\textwidth}{|p{2.5cm}|p{2.5cm}|X|X|X|X|}
\hline
\multicolumn{6}{|X|}{\textbf{Nivel: 04 }}\\\hline\textbf{Materia} & \textbf{Docente} & \textbf{Estudiantes} & \textbf{Aprobados} & \textbf{Promedio} & \textbf{\%Supera el Promedio} \\ \hline
Inmunología Clínica I & CHILA GARCIA KAREN CAROLINA & 3 & 0 & 0 & 0.00 \%\\ \hline
Lectura y Escritura Académica & PEÑA ROSAS GLORIA DEL VALLE & 14 & 13 & 34.86 & 64.29 \%\\ \hline
Uroanálisis y Líquidos Biológi & ZÚÑIGA SOSA EVELIN ALEXANDRA & 17 & 13 & 29.71 & 76.47 \%\\ \hline
Uroanálisis y Líquidos Biológi & ZÚÑIGA SOSA EVELIN ALEXANDRA & 11 & 0 & 0 & 0.00 \%\\ \hline
Uroanálisis y Líquidos Biológi & ZÚÑIGA SOSA EVELIN ALEXANDRA & 6 & 0 & 0 & 0.00 \%\\ \hline
Hematología II & RUEDA CASTILLO YAJAIRA MARILIN & 18 & 16 & 34.44 & 44.44 \%\\ \hline
Hematología II & RUEDA CASTILLO YAJAIRA MARILIN & 12 & 0 & 0 & 0.00 \%\\ \hline
Hematología II & RUEDA CASTILLO YAJAIRA MARILIN & 6 & 0 & 0 & 0.00 \%\\ \hline
Bioquimica Clínica II & HIDALGO TAPIA COSME ENRIQUE & 15 & 13 & 34.93 & 53.33 \%\\ \hline
Bioquimica Clínica II & HIDALGO TAPIA COSME ENRIQUE & 13 & 0 & 0 & 0.00 \%\\ \hline
Inmunología Clínica II & RUEDA CASTILLO YAJAIRA MARILIN & 14 & 12 & 35.07 & 42.86 \%\\ \hline
Inmunología Clínica II & RUEDA CASTILLO YAJAIRA MARILIN & 11 & 0 & 0 & 0.00 \%\\ \hline
Citología Clínica & ESTUPIÑAN SANCHEZ ALFREDO SANDINO & 13 & 0 & 0 & 0.00 \%\\ \hline
Tutorías de Acompañamiento & CHILA GARCIA KAREN CAROLINA & 20 & 0 & 0 & 0.00 \%\\ \hline
Etica Personal y Socioambienta & ZAMBRANO DUEÑAS JOSE MARIA & 9 & 9 & 44.44 & 55.56 \%\\ \hline
\end{tabularx}

\vspace{1cm}
\section{Análisis de Rendimiento}
El análisis del rendimiento académico de los estudiantes de primer nivel de TICs revela una variedad de resultados en diferentes materias. En el courses de Lectura y Escritura Académica, dirigido por la docente PEÑA ROSAS GLORIA DEL VALLE, se observa un buen desempeño, ya que de los 14 estudiantes matriculados, 13 aprobaron y el promedio del curso es de 34.86. Esto representa un porcentaje de superación del promedio del 64.29\%.

En contraste, el curso de Inmunología Clínica I, dirigido por la docente CHILA GARCIA KAREN CAROLINA, no tuvo aprobados y el promedio del curso fue de 0.00, lo que indica un bajo desempeño.

El curso de Uroanálisis y Líquidos Biológicos, dirigido por la docente ZÚÑIGA SOSA EVELIN ALEXANDRA, presentó resultados dispares entre los diferentes períodos de matrícula. Mientras que en los cursos que constaban de 17 y 11 estudiantes matriculados se alcanzaron precios de aprobación del 76.47\% y 0\% respectivamente, en los cursos con 6 estudiantes matriculados, no hubo aprobados y el promedio fue de 0.00.

En el curso de Hematología II, dirigido por la docente RUEDA CASTILLO YAJAIRA MARILIN, los resultados también fueron variados. Mientras que en los cursos que constaban de 18 y 12 estudiantes matriculados se alcanzaron precios de aprobación del 44.44\% y 0\% respectivamente, en los cursos con 6 estudiantes matriculados, no hubo aprobados y el promedio fue de 0.00.

El curso de Bioquímica Clínica II, dirigido por la docente HIDALGO TAPIA COSME ENRIQUE, presentó un buen desempeño, ya que de los 15 estudiantes matriculados, 13 aprobaron y el promedio del curso fue de 34.93, lo que representa un porcentaje de superación del promedio del 53.33\%. Sin embargo, en el curso con 13 estudiantes matriculados, no hubo aprobados y el promedio fue de 0.00.

En el curso de Inmunología Clínica II, dirigido por la docente RUEDA CASTILLO YAJAIRA MARILIN, los resultados siguen una tendencia similar, ya que de los 14 estudiantes matriculados, 12 aprobaron y el promedio del curso fue de 35.07, lo que representa un porcentaje de superación del promedio del 42.86\%. Sin embargo, en el curso con 11 estudiantes matriculados, no hubo aprobados y el promedio fue de 0.00.

El curso de Citología Clínica, dirigido por la docente ESTUPIÑAN SANCHEZ ALFREDO SANDINO, no tuvo aprobados y el promedio fue de 0.00.

El curso de Tutorías de Acompañamiento, dirigido por la docente CHILA GARCIA KAREN CAROLINA, no tuvo aprobados y el promedio fue de 0.00.

Por último, el curso de Ética Personal y Socioambiental, dirigido por la docente ZAMBRANO DUEÑAS JOSE MARIA, se destacó por un buen desempeño, ya que de los 9 estudiantes matriculados, 9 aprobaron y el promedio del curso fue de 44.44, lo que representa un porcentaje de superación del promedio del 55.56\%.

En general, los resultados del curso revelan que algunos estudiantes han logrado niveles aceptables de desempeño en algunas materias, mientras que otros han presentado dificultades significativas en otras. Es importante que las autoridades academy profundicen en la evaluación de los resultados y realicen ajustes necesarios para asegurar que los estudiantes reciban la educación de alta calidad que necesitan.\\
\vspace{1cm}\\\small
\begin{tabularx}{\textwidth}{|p{2.5cm}|p{2.5cm}|X|X|X|X|}
\hline
\multicolumn{6}{|X|}{\textbf{Nivel: 06 }}\\\hline\textbf{Materia} & \textbf{Docente} & \textbf{Estudiantes} & \textbf{Aprobados} & \textbf{Promedio} & \textbf{\%Supera el Promedio} \\ \hline
Control de Calidad I & AGREDA EGAS EYLEN AMANDA & 17 & 16 & 34.24 & 58.82 \%\\ \hline
Bacteriología Clínica II & CASTRO DEMERA DICKE ALEJANDRO & 22 & 19 & 31.41 & 54.55 \%\\ \hline
Bacteriología Clínica II & CASTRO DEMERA DICKE ALEJANDRO & 12 & 0 & 0 & 0.00 \%\\ \hline
Bacteriología Clínica II & CASTRO DEMERA DICKE ALEJANDRO & 10 & 0 & 0 & 0.00 \%\\ \hline
Bioestadística I & ACOSTA GANAN MICHAEL ANDRES & 26 & 20 & 30.65 & 57.69 \%\\ \hline
Genética y Biología Molecular & HIDALGO TAPIA COSME ENRIQUE & 29 & 29 & 35.17 & 44.83 \%\\ \hline
Diseño y Evaluación de Proyect & ESTUPIÑAN SANCHEZ ALFREDO SANDINO & 30 & 29 & 38.27 & 56.67 \%\\ \hline
Patología Clinica & CALDERON RUIZ PAOLA ALEXANDRA & 27 & 27 & 44.96 & 62.96 \%\\ \hline
Tutorías de Acompañamiento & ACOSTA GANAN MICHAEL ANDRES & 28 & 0 & 0 & 0.00 \%\\ \hline
\end{tabularx}

\vspace{1cm}
\section{Análisis de Rendimiento}
En el análisis del rendimiento académico de los estudiantes de primer nivel de TICs, se observa una variedad de resultados en cada materia. En general, se puede concluir que el promedio de rendimiento es satisfactorio, con una media global de 33.31.

La materia Bacteriología Clínica II presentó resultados desafortunados para algunos estudiantes, con una tasa de aprobación del 0\%. Esto puede indicar que hay necesidad de refinar la estrategia de enseñanza o apoyo para aquellos estudiantes que tienen dificultades con este contenido.

Por otro lado, las materias Genética y Biología Molecular y Patología Clínica presentaron resultados excelentes, con tasas de aprobación del 100\% y un promedio superior a la media global. Esto sugiere una excelencia en la enseñanza y apoyo a los estudiantes en estos temas.

La materia Diseño y Evaluación de Proyectos también mostró resultados aceptables, con un promedio de 38.27 y una tasa de aprobación del 96.67\%. Esto indica que los estudiantes tienen habilidades y conocimientos adecuados para abordar proyectos y resolver problemas.

La materia Bioestadística I presentó resultados moderados, con un promedio de 30.65 y una tasa de aprobación del 76.92\%. Esto puede indicar que los estudiantes requieren más apoyo y práctica en esta área.

Es importante destacar que la materia Tutorías de Acompañamiento no tiene un promedio y tasa de aprobación, lo que sugiere que este curso no tiene un enfoque académico tradicional, sino más bien un enfoque en apoyo y orientación a los estudiantes.

En resumen, aunque hay resultados variados en cada materia, se puede concluir que el rendimiento académico en general es satisfactorio. Sin embargo, se recomienda un enfoque en apoyo y ajustes para las materias que han mostrado resultados desafortunados, y mantener el seguimiento a las materias que han mostrado resultados excelentes.\\
\vspace{1cm}\\\small
\begin{tabularx}{\textwidth}{|p{2.5cm}|p{2.5cm}|X|X|X|X|}
\hline
\multicolumn{6}{|X|}{\textbf{Nivel: 07 }}\\\hline\textbf{Materia} & \textbf{Docente} & \textbf{Estudiantes} & \textbf{Aprobados} & \textbf{Promedio} & \textbf{\%Supera el Promedio} \\ \hline
Control de Calidad II & ZÚÑIGA SOSA EVELIN ALEXANDRA & 19 & 16 & 33.32 & 31.58 \%\\ \hline
Micología Médica & VIZCAINO ORDOÑEZ LAURA DEL ROCIO  & 22 & 20 & 33.73 & 45.45 \%\\ \hline
Micología Médica & VIZCAINO ORDOÑEZ LAURA DEL ROCIO  & 11 & 0 & 0 & 0.00 \%\\ \hline
Micología Médica & VIZCAINO ORDOÑEZ LAURA DEL ROCIO  & 11 & 0 & 0 & 0.00 \%\\ \hline
Bioestadística II & MERA CARRANZA JOSSELYN GENESIS & 16 & 15 & 40.69 & 56.25 \%\\ \hline
Prácticas Preprofesionales & ZÚÑIGA SOSA EVELIN ALEXANDRA & 22 & 0 & 0 & 0.00 \%\\ \hline
Prácticas de Servicio Comunita & ZÚÑIGA SOSA EVELIN ALEXANDRA & 12 & 12 & 45.58 & 58.33 \%\\ \hline
Enfermedades Tropicales & CASTRO DEMERA DICKE ALEJANDRO & 14 & 0 & 0 & 0.00 \%\\ \hline
Tutorías de Acompañamiento & RUEDA CASTILLO YAJAIRA MARILIN & 21 & 0 & 0 & 0.00 \%\\ \hline
Tutorías de Acompañamiento & RUEDA CASTILLO YAJAIRA MARILIN & 28 & 0 & 0 & 0.00 \%\\ \hline
\end{tabularx}

\vspace{1cm}
\section{Análisis de Rendimiento}
El presente informe destaca los resultados del rendimiento académico de los estudiantes de primer nivel de TICs en las diferentes materias impartidas en el curso. A continuación, se presentan los resultados y conclusiones obtenidos a partir de los datos proporcionados.

En cuanto a la materia de Control de Calidad II, se observó que el número de aprobados fue de 16, lo que representa un 84.21\% de los 19 estudiantes que se matricularon en la materia. El promedio del curso fue de 33.32, y el porcentaje de estudiantes que superó el promedio fue del 31.58\%. En general, se puede concluir que la materia de Control de Calidad II presentó buenos resultados, con una mayoría de estudiantes aprobados y un promedio moderado.

En el área de Micología Médica, se observaron resultados dispares en diferentes grupos de estudiantes. El primer grupo, que contó con 22 estudiantes, obtuvo un promedio de 33.73 y un porcentaje de aprobados del 45.45\%. Encontramos también otro grupo de 11 estudiantes que no aprobó la materia. A pesar de estos resultados, se puede concluir que la materia de Micología Médica es de considerable dificultad, ya que solo el 54.55\% de los estudiantes del primer grupo lo aprobó.

En la materia de Bioestadística II, se observó un buen rendimiento académico, con 15 de los 16 estudiantes aprobados (93.75\%). El promedio del curso fue de 40.69, y el porcentaje de estudiantes que superó el promedio fue del 56.25\%. Estos resultados sugieren que la materia de Bioestadística II es de moderada a alta dificultad, si bien la mayoría de los estudiantes la aprobó.

El curso de Prácticas Preprofesionales presentó un resultado negativo, ya que ninguno de los 22 estudiantes lo aprobó. Este resultado sugiere que la materia requiere un esfuerzo adicional por parte de los estudiantes.

Además, se observaron resultados similares en las materias de Prácticas de Servicio Comunitario, Enfermedades Tropicales, Tutorías de Acompañamiento y Micología Médica (grupos de 11 y 28 estudiantes), todos los cuales no aprobaron la materia.

En conclusión, el análisis del rendimiento académico de los estudiantes de primer nivel de TICs muestra una variedad de resultados. Mientras algunas materias como Bioestadística II y Prácticas de Servicio Comunitario presentaron buenos resultados, otras como Control de Calidad II y Micología Médica (grupos de 11 y 28 estudiantes) presentaron resultados más dispares. Es importante que los docentes y los responsable del curso consideren estos resultados para implementar estrategias de apoyo y mejorar el rendimiento académico de los estudiantes.\\
\vspace{1cm}\\\small
\begin{tabularx}{\textwidth}{|p{2.5cm}|p{2.5cm}|X|X|X|X|}
\hline
\multicolumn{6}{|X|}{\textbf{Nivel: 08 }}\\\hline\textbf{Materia} & \textbf{Docente} & \textbf{Estudiantes} & \textbf{Aprobados} & \textbf{Promedio} & \textbf{\%Supera el Promedio} \\ \hline
Integración Curricular & PEÑA ROSAS GLORIA DEL VALLE & 25 & 24 & 35.76 & 72.00 \%\\ \hline
Enfermedades Tropicales & CASTRO DEMERA DICKE ALEJANDRO & 26 & 26 & 34.23 & 34.62 \%\\ \hline
Enfermedades Tropicales & CASTRO DEMERA DICKE ALEJANDRO & 12 & 0 & 0 & 0.00 \%\\ \hline
Administración de Laboratorio & SALAZAR DONOSO ELIAS HUMBERTO & 15 & 15 & 44.87 & 66.67 \%\\ \hline
Toxicología & PEÑA ROSAS GLORIA DEL VALLE & 13 & 13 & 39.38 & 53.85 \%\\ \hline
Análisis y Validación de Datos & RUEDA CASTILLO YAJAIRA MARILIN & 14 & 12 & 32.64 & 57.14 \%\\ \hline
\end{tabularx}

\vspace{1cm}
\section{Análisis de Rendimiento}
El análisis del rendimiento académico del primer nivel de TICs revela una variedad de tendencias y resultados interesantes. En general, puede observarse que la cantidad de aprobados es alta en la mayoría de las materias, con un promedio general de aprobación del 75,16\% (24/25 estudiantes aprobados en Enfermedades Tropicales y 15/15 estudiantes aprobados en Administración de Laboratorio).

En cuanto al promedio del curso, se observa una tendencia hacia la excelencia, con un promedio general de 35,42 (media de los promedios de cada materia: 35,76, 34,23, 44,87, 39,38 y 32,64). Sin embargo, existen algunas materias que presentan un promedio más bajo, como Enfermedades Tropicales (34,62) y Análisis y Validación de Datos (32,64), lo que podría indicar la necesidad de enfocar esfuerzos en la docencia y el apoyo a los estudiantes en estas áreas.

De igual manera, se destaca que un porcentaje significativo de estudiantes supera el promedio en todas las materias, lo que sugiere que la mayor parte de los estudiantes han logrado alcanzar un buen desempeño académico en el curso. Sin embargo, también se observa que en algunas materias hay estudiantes que no superan el promedio, lo que podría indicar la necesidad de revisar la estrategia de enseñanza y evaluación para garantizar la comprensión y el dominio de los conceptos.

En cuanto a la integración curricular, no se pueden hacer conclusiones definitivas sin más información, pero se puede observar que no hay una clara tendencia en el rendimiento en relación con la materia impartida.

En resumen, el análisis del rendimiento académico del primer nivel de TICs revela un buen resultado global, con una alta tasa de aprobados y promedios altos en la mayoría de las materias. Sin embargo, también se observan algunas áreas en las que es necesario enfocar esfuerzos para asegurar el logro de los objetivos del curso.\\
\vspace{1cm}\\\begin{tabularx}{\textwidth}{|X|X|X|}
    \hline
    \textbf{ELABORADO POR:} & \textbf{REVISADO POR:} & \textbf{APROBADO POR:} \\ \hline
    Firma: & Firma: & Firma:\\
    &&\\
    &&\\
    &&\\ \hline
    \textbf{Nombre: Homero Velasteguí} & \textbf{Nombre: Manuel Nevarez} & \textbf{Nombre: Pablo Pico Valencia PhD.} \\ \hline
    \textbf{Cargo: Coordinador Carrera} & \textbf{Cargo: Consejo de Escuela} & \textbf{Cargo: Director Académico} \\ \hline
    \textbf{Fecha: 9/3/2024} & \textbf{Fecha: 9/3/2024} & \textbf{Fecha: 9/3/2024} \\ \hline
    \end{tabularx}
