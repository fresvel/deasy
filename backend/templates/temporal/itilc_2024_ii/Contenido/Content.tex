\small
\begin{tabularx}{\textwidth}{|p{2.5cm}|p{2.5cm}|X|X|X|X|}
\hline
\multicolumn{6}{|X|}{\textbf{Nivel: 02 }}\\\hline\textbf{Materia} & \textbf{Docente} & \textbf{Estudiantes} & \textbf{Aprobados} & \textbf{Promedio} & \textbf{\%Supera el Promedio} \\ \hline
Química General & PEÑA ROSAS GLORIA DEL VALLE & 12 & 0 & 0 & 0.00 \%\\ \hline
Química Analítica & PEÑA ROSAS GLORIA DEL VALLE & 16 & 13 & 32.25 & 56.25 \%\\ \hline
Química Analítica & PEÑA ROSAS GLORIA DEL VALLE & 9 & 0 & 0 & 0.00 \%\\ \hline
Química Analítica & PEÑA ROSAS GLORIA DEL VALLE & 7 & 0 & 0 & 0.00 \%\\ \hline
Quimica Organica & ACOSTA GANAN MICHAEL ANDRES & 19 & 13 & 30.26 & 47.37 \%\\ \hline
Quimica Organica & ACOSTA GANAN MICHAEL ANDRES & 11 & 0 & 0 & 0.00 \%\\ \hline
Quimica Organica & ACOSTA GANAN MICHAEL ANDRES & 8 & 0 & 0 & 0.00 \%\\ \hline
Bioquimica & ACOSTA GANAN MICHAEL ANDRES & 16 & 8 & 30.5 & 43.75 \%\\ \hline
Bioquimica & ACOSTA GANAN MICHAEL ANDRES & 10 & 0 & 0 & 0.00 \%\\ \hline
Microbiología & CASTRO DEMERA DICKE ALEJANDRO & 19 & 8 & 24.47 & 52.63 \%\\ \hline
Microbiología & CASTRO DEMERA DICKE ALEJANDRO & 13 & 0 & 0 & 0.00 \%\\ \hline
Microbiología & CASTRO DEMERA DICKE ALEJANDRO & 6 & 0 & 0 & 0.00 \%\\ \hline
Tutorías de Acompañamiento & PEÑA ROSAS GLORIA DEL VALLE & 18 & 0 & 0 & 0.00 \%\\ \hline
Fundamentos de la Investigació & SUAREZ LEZCANO JOSE  & 12 & 9 & 39.33 & 75.00 \%\\ \hline
\end{tabularx}

\vspace{1cm}
\section{Análisis de Rendimiento}
En el análisis del rendimiento académico de los estudiantes de primer nivel de TICs, se observa que la materia que ha obtenido el mejor promedio es la "Fundamentos de la Investigación" con un promedio de 39.33, correspondiendo a un 75\% de aprobados. Esta materia se encuentra a un nivel mucho más alto que la mayoría de las demás, lo que sugiere que los estudiantes han logrado una firme comprensión de los conceptos y habilidades relacionados con la investigación.

En contraste, se halla la materia de "Química General" y "Química Analítica" que han obtenido un promedio de 0, lo que indica que no hay estudiantes aprobados en estas asignaturas. Esto puede ser un indicio de que los estudiantes necesitan mayor apoyo y retroalimentación en estos temas.

Las materias de "Quimica Organica" y "Bioquimica" también han presentado un rendimiento mediocrero, con promedios de 30.26 y 30.5 respectivamente, y un porcentaje de aprobados entre el 43.75\% y el 47.37\%. Estas materias pueden requerir una mayor atención y revisión por parte de los profesores y los estudiantes para asegurar un mejor entendiimiento de las conceptos.

Finalmente, la materia de "Microbiología" ha obtenido un promedio de 24.47, correspondiendo a un 52.63\% de aprobados, lo que sugiere que los estudiantes pueden necesitar mayor apoyo y retroalimentación en este tema.

En cuanto a las tutorías de acompañamiento, no hay aprobados, lo que sugiere que estos estudiantes pueden necesitar más apoyo y orientación para lograr un mayor éxito académico.

En resumen, el análisis del rendimiento académico de los estudiantes de primer nivel de TICs muestra que la materia "Fundamentos de la Investigación" ha obtenido el mejor promedio, mientras que las materias de "Química General" y "Química Analítica" han presentado un rendimiento pobre. Es importante que los profesores y los estudiantes tengan en cuenta estos resultados para identificar áreas de fortaleza y debilidad y trabajar en mejorar el rendimiento académico.\\
\vspace{1cm}\\\small
\begin{tabularx}{\textwidth}{|p{2.5cm}|p{2.5cm}|X|X|X|X|}
\hline
\multicolumn{6}{|X|}{\textbf{Nivel: 01 }}\\\hline\textbf{Materia} & \textbf{Docente} & \textbf{Estudiantes} & \textbf{Aprobados} & \textbf{Promedio} & \textbf{\%Supera el Promedio} \\ \hline
Química General & PEÑA ROSAS GLORIA DEL VALLE & 26 & 23 & 33.42 & 57.69 \%\\ \hline
Química General & PEÑA ROSAS GLORIA DEL VALLE & 14 & 0 & 0 & 0.00 \%\\ \hline
Citología e Histología & RUEDA CASTILLO YAJAIRA MARILIN & 26 & 22 & 33.88 & 50.00 \%\\ \hline
Citología e Histología & RUEDA CASTILLO YAJAIRA MARILIN & 15 & 0 & 0 & 0.00 \%\\ \hline
Citología e Histología & RUEDA CASTILLO YAJAIRA MARILIN & 11 & 0 & 0 & 0.00 \%\\ \hline
Anatomía y Fisiología & ESTUPIÑAN SANCHEZ ALFREDO SANDINO & 25 & 24 & 40.32 & 60.00 \%\\ \hline
Anatomía y Fisiología & ESTUPIÑAN SANCHEZ ALFREDO SANDINO & 14 & 0 & 0 & 0.00 \%\\ \hline
Anatomía y Fisiología & ESTUPIÑAN SANCHEZ ALFREDO SANDINO & 11 & 0 & 0 & 0.00 \%\\ \hline
Técnicas Básicas de Laboratori & HIDALGO TAPIA COSME ENRIQUE & 14 & 14 & 34 & 42.86 \%\\ \hline
Técnicas Básicas de Laboratori & HIDALGO TAPIA COSME ENRIQUE & 11 & 10 & 32.91 & 72.73 \%\\ \hline
Tutorías de Acompañamiento & PEÑA ROSAS GLORIA DEL VALLE & 26 & 0 & 0 & 0.00 \%\\ \hline
Tecnologías de la Información  & RIVERA BONE CARLOS ALEJANDRO & 25 & 22 & 41.16 & 80.00 \%\\ \hline
Comunicación Oral y Escrita & CABALLERO MOREIRA JAIRON ADRIAN & 25 & 23 & 39.76 & 64.00 \%\\ \hline
\end{tabularx}

\vspace{1cm}
\section{Análisis de Rendimiento}
Análisis del rendimiento académico del primer nivel de TICs

El análisis del rendimiento académico revela que los estudiantes han demostrado un desempeño diverso en cada materia. En Química General, si bien el número de estudiante es mayor, la proporción de aprobados es significativamente baja, con un 77,67\% de aprobados. Por otro lado, en Citología e Histología, la tasa de aprobados es más baja, con solo un 15,38\% de aprobados. En Anatomía y Fisiología, se observa un desempeño excelente, con un 96\% de aprobados.

En Técnicas Básicas de Laboratorio, se destaca el alto porcentaje de aprobados, con un 72,73\% de aprobados, lo que sugiere que los estudiantes han demostrado una buena comprensión de los conceptos y habilidades laboratorios. En Tutorías de Acompañamiento, todos los cursos resultaron con un cero de aprobados, lo que sugiere que los estudiantes no han podido aprovechar al máximo este tipo de apoyo académico.

En Tecnologías de la Información, el curso ha demostrado un buen rendimiento, con un 80\% de aprobados. En Comunicación Oral y Escrita, el curso ha demostrado un desempeño regular, con un 64\% de aprobados.

En términos generales, se puede concluir que los estudiantes tienen un rendimiento académico diverso en cada materia, con algunas materias demostrando un desempeño excelente y otras mostrando necesidad de mejora. Es importante que los docentes y los administradores académicos analicen estos resultados para identificar áreas de fortaleza y debilidad y desarrollen estrategias para apoyar el éxito de los estudiantes.

Es importante destacar que el promedio del curso es de 34.67, lo que sugiere que en promedio los estudiantes han demostrado un buen desempeño. Sin embargo, es importante recordar que el rendimiento académico no solo depende de la figura del estudiante, sino también del apoyo que recibe y de la calidad de la educación que se brinda.\\
\vspace{1cm}\\\small
\begin{tabularx}{\textwidth}{|p{2.5cm}|p{2.5cm}|X|X|X|X|}
\hline
\multicolumn{6}{|X|}{\textbf{Nivel: 05 }}\\\hline\textbf{Materia} & \textbf{Docente} & \textbf{Estudiantes} & \textbf{Aprobados} & \textbf{Promedio} & \textbf{\%Supera el Promedio} \\ \hline
Bioquimica Clínica II & HIDALGO TAPIA COSME ENRIQUE & 2 & 0 & 0 & 0.00 \%\\ \hline
Inmunología Clínica II & RUEDA CASTILLO YAJAIRA MARILIN & 3 & 0 & 0 & 0.00 \%\\ \hline
Bacteriología Clínica I & CASTRO DEMERA DICKE ALEJANDRO & 32 & 20 & 28.13 & 62.50 \%\\ \hline
Bacteriología Clínica I & CASTRO DEMERA DICKE ALEJANDRO & 14 & 0 & 0 & 0.00 \%\\ \hline
Bacteriología Clínica I & CASTRO DEMERA DICKE ALEJANDRO & 18 & 0 & 0 & 0.00 \%\\ \hline
Epidemiología y Realidad Nacio & ACOSTA GANAN MICHAEL ANDRES & 26 & 17 & 30.88 & 53.85 \%\\ \hline
Virología & ESTUPIÑAN SANCHEZ ALFREDO SANDINO & 22 & 22 & 41.77 & 63.64 \%\\ \hline
Bioquímica Clínica III & ZÚÑIGA SOSA EVELIN ALEXANDRA & 26 & 23 & 32.85 & 57.69 \%\\ \hline
Bioquímica Clínica III & ZÚÑIGA SOSA EVELIN ALEXANDRA & 13 & 0 & 0 & 0.00 \%\\ \hline
Bioquímica Clínica III & ZÚÑIGA SOSA EVELIN ALEXANDRA & 13 & 0 & 0 & 0.00 \%\\ \hline
Inmunohematología & CHILA GARCIA KAREN CAROLINA & 39 & 33 & 30.69 & 61.54 \%\\ \hline
Inmunohematología & CHILA GARCIA KAREN CAROLINA & 13 & 0 & 0 & 0.00 \%\\ \hline
Inmunohematología & CHILA GARCIA KAREN CAROLINA & 16 & 0 & 0 & 0.00 \%\\ \hline
Citología Clínica & ESTUPIÑAN SANCHEZ ALFREDO SANDINO & 25 & 25 & 43.16 & 48.00 \%\\ \hline
Citología Clínica & ESTUPIÑAN SANCHEZ ALFREDO SANDINO & 12 & 0 & 0 & 0.00 \%\\ \hline
Tutorías de Acompañamiento & ACOSTA GANAN MICHAEL ANDRES & 32 & 0 & 0 & 0.00 \%\\ \hline
Inmunohematología & CHILA GARCIA KAREN CAROLINA & 10 & 0 & 0 & 0.00 \%\\ \hline
\end{tabularx}

\vspace{1cm}
\section{Análisis de Rendimiento}
El análisis del rendimiento académico de los estudiantes del primer nivel de TICs revela patrones interesantes y desafíos en diferentes materias. En general, se observa que la mayoría de las materias tienen al menos un grupo de estudiantes que presenta un rendimiento destacado, con promedios superiores a 30 puntos.

Sin embargo, también es relevante destacar que hay algunas materias en las que el rendimiento es más desigual, con un gran número de estudiantes que no han aprobado el curso. Esto es el caso de "Bioquimica Clínica II", "Inmunología Clínica II", "Bacteriología Clínica I" (con algunos grupos) y "Tutorías de Acompañamiento", en las que no ha habido aprobados.

Por otro lado, las materias en que se observa un rendimiento más homogéneo y alto son "Bacteriología Clínica I" (con un grupo que tiene un promedio de 28.13), "Virología" (con un promedio de 41.77), "Bioquímica Clínica III" (con un promedio de 32.85 en un grupo) y "Inmunohematología" (con un promedio de 30.69 en diferentes grupos). Es digno de mencionar que en estas materias, más del 50\% de los estudiantes han logrado aprobar el curso.

En cuanto al porcentaje de estudiantes que superan el promedio, se puede observar que solo en algunos casos se llega a cifras significativas, como en "Virología" (63.64\%) y "Bacteriología Clínica I" (62.50\%). En otras materias, este porcentaje es más bajo, como en "Bioquímica Clínica III" (57.69\%) y "Inmunohematología" (61.54\%).

En conclusión, el análisis del rendimiento académico de los estudiantes del primer nivel de TICs revela que hay materias en las que el rendimiento es más desigual y materiales en las que se observa un rendimiento más homogéneo y alto. Es importante que los docentes y administradores involucrados en el curso analizen y aborden estas desigualdades para brindar apoyo y recursos a los estudiantes que lo necesitan, con el fin de mejorar el rendimiento y la retainibilidad de los estudiantes.\\
\vspace{1cm}\\\small
\begin{tabularx}{\textwidth}{|p{2.5cm}|p{2.5cm}|X|X|X|X|}
\hline
\multicolumn{6}{|X|}{\textbf{Nivel: 03 }}\\\hline\textbf{Materia} & \textbf{Docente} & \textbf{Estudiantes} & \textbf{Aprobados} & \textbf{Promedio} & \textbf{\%Supera el Promedio} \\ \hline
Bioquimica & ACOSTA GANAN MICHAEL ANDRES & 6 & 0 & 0 & 0.00 \%\\ \hline
Parasitología & CHILA GARCIA KAREN CAROLINA & 19 & 12 & 28.53 & 63.16 \%\\ \hline
Parasitología & CHILA GARCIA KAREN CAROLINA & 9 & 0 & 0 & 0.00 \%\\ \hline
Parasitología & CHILA GARCIA KAREN CAROLINA & 10 & 0 & 0 & 0.00 \%\\ \hline
Análisis Instrumental & ACOSTA GANAN MICHAEL ANDRES & 14 & 8 & 27.79 & 57.14 \%\\ \hline
Análisis Instrumental & ACOSTA GANAN MICHAEL ANDRES & 9 & 8 & 32.78 & 44.44 \%\\ \hline
Inmunología Clínica I & CHILA GARCIA KAREN CAROLINA & 25 & 16 & 29.44 & 64.00 \%\\ \hline
Inmunología Clínica I & CHILA GARCIA KAREN CAROLINA & 10 & 0 & 0 & 0.00 \%\\ \hline
Inmunología Clínica I & CHILA GARCIA KAREN CAROLINA & 12 & 0 & 0 & 0.00 \%\\ \hline
Hematología I & AGREDA EGAS EYLEN AMANDA & 18 & 12 & 29.39 & 66.67 \%\\ \hline
Hematología I & AGREDA EGAS EYLEN AMANDA & 8 & 0 & 0 & 0.00 \%\\ \hline
Hematología I & AGREDA EGAS EYLEN AMANDA & 7 & 0 & 0 & 0.00 \%\\ \hline
Hematología I & AGREDA EGAS EYLEN AMANDA & 3 & 0 & 0 & 0.00 \%\\ \hline
Bioquímica Clínica I & HIDALGO TAPIA COSME ENRIQUE & 20 & 18 & 32.15 & 50.00 \%\\ \hline
Bioquímica Clínica I & HIDALGO TAPIA COSME ENRIQUE & 17 & 0 & 0 & 0.00 \%\\ \hline
Bioquímica Clínica I & HIDALGO TAPIA COSME ENRIQUE & 3 & 0 & 0 & 0.00 \%\\ \hline
Tutorías de Acompañamiento & CHILA GARCIA KAREN CAROLINA & 23 & 0 & 0 & 0.00 \%\\ \hline
Contextos e Interculturalidad & BAUTISTA MEJIA KATIUSKA MARIA & 21 & 18 & 38.1 & 61.90 \%\\ \hline
Jesucristo y la Persona de Hoy & QUINTERO ROSALES FRANCISCO JHONNY & 12 & 7 & 30.5 & 41.67 \%\\ \hline
\end{tabularx}

\vspace{1cm}
\section{Análisis de Rendimiento}
El análisis del rendimiento académico de los estudiantes de primer nivel de TICs revela una diversidad de resultados dentro de las diferentes asignaturas. En cuanto a la cantidad de estudiantes matriculados, se puede observar que algunas asignaturas tienen un gran número de estudiantes (como Parasitología y Inmunología Clínica I, con 19 y 25 estudiantes respectivamente), mientras que otras tienen un número más reducido (como Bioquímica, con 6 estudiantes).

En cuanto al rendimiento académico, se destaca que algunas asignaturas tienen un porcentaje de aprobados significativo. Por ejemplo, en Parasitología la tasa de aprobación es del 63.16\%, lo que sugiere que la mayoría de los estudiantes están bien preparados para enfrentar los desafíos de esta materia. De manera similar, en Inmunología Clínica I la tasa de aprobación es del 64\%, lo que indica una buena comprensión de los conceptos fundamentales de la asignatura.

Sin embargo, también se observan asignaturas con tasas de aprobación más bajas. Por ejemplo, en Análisis Instrumental y Bioquímica Clínica I la tasa de aprobación es del 44.44\% y 50\% respectivamente, lo que sugiere que algunos estudiantes pueden necesitar más apoyo y estructura para lograr un buen rendimiento.

En cuanto al promedio del curso, se observa una tendencia general de promedios relativamente altos, especialmente en asignaturas como Inmunología Clínica I y Parasitología, que alcanzan valores de 29.44 y 28.53 respectivamente. Sin embargo, también se observan asignaturas con promedios más bajos, como Bioquímica Clínica I y Análisis Instrumental, que alcanzan valores de 32.15 y 27.79 respectivamente.

En general, el análisis del rendimiento académico de los estudiantes de primer nivel de TICs sugiere que necesitan una mayor atención y estructura en algunas asignaturas, pero también hay muestra de un buen desempeño académico en las asignaturas que requieren un mayor esfuerzo y comprensión. Es importante que los docentes y administradores académicos trabajen juntos para identificar las áreas de necesidad y brindar el apoyo necesario para que los estudiantes puedan lograr un buen rendimiento y éxito en sus estudios.\\
\vspace{1cm}\\\small
\begin{tabularx}{\textwidth}{|p{2.5cm}|p{2.5cm}|X|X|X|X|}
\hline
\multicolumn{6}{|X|}{\textbf{Nivel: 04 }}\\\hline\textbf{Materia} & \textbf{Docente} & \textbf{Estudiantes} & \textbf{Aprobados} & \textbf{Promedio} & \textbf{\%Supera el Promedio} \\ \hline
Inmunología Clínica I & CHILA GARCIA KAREN CAROLINA & 3 & 0 & 0 & 0.00 \%\\ \hline
Lectura y Escritura Académica & PEÑA ROSAS GLORIA DEL VALLE & 14 & 13 & 34.86 & 64.29 \%\\ \hline
Uroanálisis y Líquidos Biológi & ZÚÑIGA SOSA EVELIN ALEXANDRA & 17 & 13 & 29.71 & 76.47 \%\\ \hline
Uroanálisis y Líquidos Biológi & ZÚÑIGA SOSA EVELIN ALEXANDRA & 11 & 0 & 0 & 0.00 \%\\ \hline
Uroanálisis y Líquidos Biológi & ZÚÑIGA SOSA EVELIN ALEXANDRA & 6 & 0 & 0 & 0.00 \%\\ \hline
Hematología II & RUEDA CASTILLO YAJAIRA MARILIN & 18 & 16 & 34.44 & 44.44 \%\\ \hline
Hematología II & RUEDA CASTILLO YAJAIRA MARILIN & 12 & 0 & 0 & 0.00 \%\\ \hline
Hematología II & RUEDA CASTILLO YAJAIRA MARILIN & 6 & 0 & 0 & 0.00 \%\\ \hline
Bioquimica Clínica II & HIDALGO TAPIA COSME ENRIQUE & 15 & 13 & 34.93 & 53.33 \%\\ \hline
Bioquimica Clínica II & HIDALGO TAPIA COSME ENRIQUE & 13 & 0 & 0 & 0.00 \%\\ \hline
Inmunología Clínica II & RUEDA CASTILLO YAJAIRA MARILIN & 14 & 12 & 35.07 & 42.86 \%\\ \hline
Inmunología Clínica II & RUEDA CASTILLO YAJAIRA MARILIN & 11 & 0 & 0 & 0.00 \%\\ \hline
Citología Clínica & ESTUPIÑAN SANCHEZ ALFREDO SANDINO & 13 & 0 & 0 & 0.00 \%\\ \hline
Tutorías de Acompañamiento & CHILA GARCIA KAREN CAROLINA & 20 & 0 & 0 & 0.00 \%\\ \hline
Etica Personal y Socioambienta & ZAMBRANO DUEÑAS JOSE MARIA & 9 & 9 & 44.44 & 55.56 \%\\ \hline
\end{tabularx}

\vspace{1cm}
\section{Análisis de Rendimiento}
En el análisis de los resultados del rendimiento académico, se observa que la materia con el mayor número de estudiantes matriculados es "Lectura y Escritura Académica" con 14 estudiantes, seguida de "Hematología II" con 18 estudiantes. Por otro lado, la materia con el menor número de estudiantes es "Citología Clínica" y "Tutorías de Acompañamiento" con solo 13 estudiantes cada uno.

En cuanto al número de aprobados, "Lectura y Escritura Académica" presenta el mayor número de estudiantes aprobados, con 13 estudiante (64.29\%). La materia "Uroanálisis y Líquidos Biológi" y "Hematología II" también presentan un buen desempeño, con 13 aprobados en ambos casos, representando el 76.47\% y el 44.44\%, respectivamente.

En cuanto al promedio del curso, se observa que las materias "Lectura y Escritura Académica", "Hematología II" y "Bioquimica Clínica II" tienen un promedio superior a 30, con 34.86, 34.44 y 34.93, respectivamente. Por otro lado, las materias "Inmunología Clínica I" y "Inmunología Clínica II" tienen un promedio más bajo, con 0.00 y 35.07, respectivamente.

Finalmente, el porcentaje de estudiantes que superan el promedio es relativamente bajo en algunas materias, como "Uroanálisis y Líquidos Biológi" y "Inmunología Clínica II", donde solo un pequeño porcentaje de estudiantes logra superar el promedio. Por otro lado, en materias como "Lectura y Escritura Académica" y "Hematología II", más del 50\% de los estudiantes superan el promedio.

En resumen, el análisis de los resultados del rendimiento académico revela que las materias con mejor desempeño son "Lectura y Escritura Académica", "Uroanálisis y Líquidos Biológi" y "Hematología II", mientras que las materias con peor desempeño son "Inmunología Clínica I", "Inmunología Clínica II" y "Citología Clínica". Es importante considerar estas observaciones para implementar cambios en el currículo y la enseñanza para mejorar el rendimiento académico de los estudiantes.\\
\vspace{1cm}\\\small
\begin{tabularx}{\textwidth}{|p{2.5cm}|p{2.5cm}|X|X|X|X|}
\hline
\multicolumn{6}{|X|}{\textbf{Nivel: 06 }}\\\hline\textbf{Materia} & \textbf{Docente} & \textbf{Estudiantes} & \textbf{Aprobados} & \textbf{Promedio} & \textbf{\%Supera el Promedio} \\ \hline
Control de Calidad I & AGREDA EGAS EYLEN AMANDA & 17 & 16 & 34.24 & 58.82 \%\\ \hline
Bacteriología Clínica II & CASTRO DEMERA DICKE ALEJANDRO & 22 & 19 & 31.41 & 54.55 \%\\ \hline
Bacteriología Clínica II & CASTRO DEMERA DICKE ALEJANDRO & 12 & 0 & 0 & 0.00 \%\\ \hline
Bacteriología Clínica II & CASTRO DEMERA DICKE ALEJANDRO & 10 & 0 & 0 & 0.00 \%\\ \hline
Bioestadística I & ACOSTA GANAN MICHAEL ANDRES & 26 & 20 & 30.65 & 57.69 \%\\ \hline
Genética y Biología Molecular & HIDALGO TAPIA COSME ENRIQUE & 29 & 29 & 35.17 & 44.83 \%\\ \hline
Diseño y Evaluación de Proyect & ESTUPIÑAN SANCHEZ ALFREDO SANDINO & 30 & 29 & 38.27 & 56.67 \%\\ \hline
Patología Clinica & CALDERON RUIZ PAOLA ALEXANDRA & 27 & 27 & 44.96 & 62.96 \%\\ \hline
Tutorías de Acompañamiento & ACOSTA GANAN MICHAEL ANDRES & 28 & 0 & 0 & 0.00 \%\\ \hline
\end{tabularx}

\vspace{1cm}
\section{Análisis de Rendimiento}
Análisis del rendimiento académico del curso de primer nivel de TICs

El análisis de los datos presentados revela una variedad de resultados en el rendimiento académico de los estudiantes del primer nivel de TICs. En general, observamos que la mayoría de los estudiantes presentan una frecuencia moderada en la obtención de aprobados, con un promedio general de 54.67\%.

En la materia de Bacteriología Clínica II, se observa una tendencia negativa en la evolución de los resultados, con una nota promedio de 31.41 y un porcentaje de aprobados del 54.55. Sin embargo, esta tendencia se repite en la misma materia, con los cursos 12 y 10, que no tienen aprobados y presentan una nota promedio de 0.00.

Por otro lado, materias como Diseño y Evaluación de Proyectos y Patología Clínica presentan un rendimiento satisfactorio, con una nota promedio de 38.27 y 44.96, respectivamente, y un porcentaje de aprobados del 56.67\% y 62.96\%.

En cuanto a las materias de Bacteriología Clínica II y Tutorías de Acompañamiento, presentan un rendimiento pobre, con una nota promedio de 0.00 y sin aprobados, respectivamente.

En la materia de Bioestadística I, se observa un rendimiento moderado, con una nota promedio de 30.65 y un porcentaje de aprobados del 57.69\%.

Por último, la materia de Genética y Biología Molecular presenta un rendimiento inferior, con una nota promedio de 35.17 y un porcentaje de aprobados del 44.83\%.

En conclusión, el análisis de los datos revela una diversidad de resultados en el rendimiento académico de los estudiantes del primer nivel de TICs. A pesar de que algunas materias presentan un rendimiento satisfactorio, otras presentan un rendimiento pobre. Es importante que se realicen esfuerzos para mejorar la coordinación y el seguimiento de los programas de estudio, y para brindar apoyo a los estudiantes que necesitan mayor atención.\\
\vspace{1cm}\\\small
\begin{tabularx}{\textwidth}{|p{2.5cm}|p{2.5cm}|X|X|X|X|}
\hline
\multicolumn{6}{|X|}{\textbf{Nivel: 07 }}\\\hline\textbf{Materia} & \textbf{Docente} & \textbf{Estudiantes} & \textbf{Aprobados} & \textbf{Promedio} & \textbf{\%Supera el Promedio} \\ \hline
Control de Calidad II & ZÚÑIGA SOSA EVELIN ALEXANDRA & 19 & 16 & 33.32 & 31.58 \%\\ \hline
Micología Médica & VIZCAINO ORDOÑEZ LAURA DEL ROCIO  & 22 & 20 & 33.73 & 45.45 \%\\ \hline
Micología Médica & VIZCAINO ORDOÑEZ LAURA DEL ROCIO  & 11 & 0 & 0 & 0.00 \%\\ \hline
Micología Médica & VIZCAINO ORDOÑEZ LAURA DEL ROCIO  & 11 & 0 & 0 & 0.00 \%\\ \hline
Bioestadística II & MERA CARRANZA JOSSELYN GENESIS & 16 & 15 & 40.69 & 56.25 \%\\ \hline
Prácticas Preprofesionales & ZÚÑIGA SOSA EVELIN ALEXANDRA & 22 & 0 & 0 & 0.00 \%\\ \hline
Prácticas de Servicio Comunita & ZÚÑIGA SOSA EVELIN ALEXANDRA & 12 & 12 & 45.58 & 58.33 \%\\ \hline
Enfermedades Tropicales & CASTRO DEMERA DICKE ALEJANDRO & 14 & 0 & 0 & 0.00 \%\\ \hline
Tutorías de Acompañamiento & RUEDA CASTILLO YAJAIRA MARILIN & 21 & 0 & 0 & 0.00 \%\\ \hline
Tutorías de Acompañamiento & RUEDA CASTILLO YAJAIRA MARILIN & 28 & 0 & 0 & 0.00 \%\\ \hline
\end{tabularx}

\vspace{1cm}
\section{Análisis de Rendimiento}
Análisis del Rendimiento Académico del Primer Nivel de TICs

El presente análisis se centró en examinar el rendimiento académico de los estudiantes del primer nivel de TICs, categorizando las materias en función de su período lectivo. Se observó que el curso "Control de Calidad II", impartido por la docente Zúñiga Sosa Evelin Alexandra, cuenta con un número de 19 matriculados y un promedio del curso del 33.32\%, con un resultado de aprobados de 16 estudiantes, lo que representa un porcentaje de superación del promedio del 31.58\%.

En cuanto al curso "Micología Médica", impartido por la docente Vizcaino Ordóñez Laura del Rocío, se observó que hay más de un grupo con diferente número de matriculados y resultados. El grupo más grande contó con 22 estudiantes, un promedio del curso del 33.73\%, y un porcentaje de aprobados del 45.45\%. En contraste, los demás grupos presentaron resultados nefastos, con un promedio del curso de 0 y un porcentaje de aprobados del 0\%.

El curso "Bioestadística II", impartido por Mera Carranza Joselyn Genesis, tuvo un mejor desempeño, con un promedio del curso del 40.69\% y un porcentaje de aprobados del 56.25\%. Sin embargo, los cursos "Prácticas Preprofesionales" y "Prácticas de Servicio Comunitario", ambos impartidos por Zúñiga Sosa Evelin Alexandra, no presentaron aprobados, lo que indica un déficit en la capacidad de aplicación y enfrentamiento de situaciones reales por parte de los estudiantes.

Por otro lado, el curso "Enfermedades Tropicales", impartido por Castro Demera Dicke Alejandro, no contó con aprobados, lo que sugiere una necesidad de refuerzo en el aprendizaje de contenidos básicos en salud. De igual manera, los cursos "Tutorías de Acompañamiento", impartidos por Rueda Castillo Yajaira Marilin, no presentaron aprobados, lo que puede indicar una falta de comprensión o aplicación de los conceptos impartidos en el curso.

En general, se puede observar que hay materias que tienen un mejor desempeño que otras, lo que puede deberse a la capacitación previa de los estudiantes, la calidad de los contenidos impartidos o la efectividad de los métodos de enseñanza. Es importante que las instituciones educativas tomen medidas para mejorar la calidad de la educación y el apoyo brindado a los estudiantes, para que puedan superar los déficits y alcanzar un mejor rendimiento académico.\\
\vspace{1cm}\\\small
\begin{tabularx}{\textwidth}{|p{2.5cm}|p{2.5cm}|X|X|X|X|}
\hline
\multicolumn{6}{|X|}{\textbf{Nivel: 08 }}\\\hline\textbf{Materia} & \textbf{Docente} & \textbf{Estudiantes} & \textbf{Aprobados} & \textbf{Promedio} & \textbf{\%Supera el Promedio} \\ \hline
Integración Curricular & PEÑA ROSAS GLORIA DEL VALLE & 25 & 24 & 35.76 & 72.00 \%\\ \hline
Enfermedades Tropicales & CASTRO DEMERA DICKE ALEJANDRO & 26 & 26 & 34.23 & 34.62 \%\\ \hline
Enfermedades Tropicales & CASTRO DEMERA DICKE ALEJANDRO & 12 & 0 & 0 & 0.00 \%\\ \hline
Administración de Laboratorio & SALAZAR DONOSO ELIAS HUMBERTO & 15 & 15 & 44.87 & 66.67 \%\\ \hline
Toxicología & PEÑA ROSAS GLORIA DEL VALLE & 13 & 13 & 39.38 & 53.85 \%\\ \hline
Análisis y Validación de Datos & RUEDA CASTILLO YAJAIRA MARILIN & 14 & 12 & 32.64 & 57.14 \%\\ \hline
\end{tabularx}

\vspace{1cm}
\section{Análisis de Rendimiento}
Después de analizar la información de rendimiento académico de los estudiantes de primer nivel de TICs, se puede observar que el curso ha mostrado resultados variados en diferentes materias.

En la materia de Enfermedades Tropicales, con un total de 38 estudiantes matriculados (dos cursos diferentes), se pueden observar dos resultados encontrados. El curso con 26 estudiantes ha logrado un mejor rendimiento, con un promedio de 34.62 y un porcentaje de aprobados del 96.15\%. Por otro lado, el curso con 12 estudiantes ha obtenido un resultado difícil de superar, con un promedio de 0.00 y solo un estudiante que ha aprobado, lo que representa un porcentaje de aprobados del 0.00\%.

En la materia de Administración de Laboratorio, se han matriculado 15 estudiantes, quienes han logrado un promedio de 44.87 y un porcentaje de aprobados del 100.00\%. Esta materia ha sido una de las más exitosas en términos de rendimiento académico.

En Toxicología, se han matriculado 13 estudiantes, quienes han logrado un promedio de 39.38 y un porcentaje de aprobados del 100.00\%. Igualmente, esta materia ha tenido un buen desempeño en términos de rendimiento académico.

Por último, en la materia de Análisis y Validación de Datos, se han matriculado 14 estudiantes, quienes han logrado un promedio de 32.64 y un porcentaje de aprobados del 85.71\%. Aunque no ha sido la materia con mejor rendimiento, se puede observar que la mayoría de los estudiantes han aprobado.

En cuanto a la integración curricular, se puede observar que algunas materias, como Enfermedades Tropicales y Toxicología, han logrado mejores resultados que otras, como Análisis y Validación de Datos. Estos resultados pueden ser influenciados por factores como la calidad de la enseñanza, la motivación de los estudiantes y la relevancia de la materia en la carrera de TICs.

En general, se puede concluir que el curso de primer nivel de TICs ha mostrado resultados variados en diferentes materias. Algunas materias han logrado buenos resultados, mientras que otras han necesitado más esfuerzo y atención. Es importante que los docentes y los estudiantes realicen un análisis detallado de los resultados y trabajen juntos para mejorar el rendimiento académico y la integración curricular.\\
\vspace{1cm}\\\begin{tabularx}{\textwidth}{|X|X|X|}
    \hline
    \textbf{ELABORADO POR:} & \textbf{REVISADO POR:} & \textbf{APROBADO POR:} \\ \hline
    Firma: & Firma: & Firma:\\
    &&\\
    &&\\
    &&\\ \hline
    \textbf{Nombre: Homero Velasteguí} & \textbf{Nombre: Manuel Nevarez} & \textbf{Nombre: Pablo Pico Valencia PhD.} \\ \hline
    \textbf{Cargo: Coordinador Carrera} & \textbf{Cargo: Consejo de Escuela} & \textbf{Cargo: Director Académico} \\ \hline
    \textbf{Fecha: 9/3/2024} & \textbf{Fecha: 9/3/2024} & \textbf{Fecha: 9/3/2024} \\ \hline
    \end{tabularx}
