\small
\begin{tabularx}{\textwidth}{|p{2.5cm}|p{2.5cm}|X|X|X|X|}
\hline
\multicolumn{6}{|X|}{\textbf{Nivel: 02 }}\\\hline\textbf{Materia} & \textbf{Docente} & \textbf{Estudiantes} & \textbf{Aprobados} & \textbf{Promedio} & \textbf{\%Supera el Promedio} \\ \hline
Microbiología & CASTRO DEMERA DICKE ALEJANDRO & 6 & 0 & 0 & 0.00 \%\\ \hline
Tutorías de Acompañamiento & PEÑA ROSAS GLORIA DEL VALLE & 18 & 0 & 0 & 0.00 \%\\ \hline
Fundamentos de la Investigació & SUAREZ LEZCANO JOSE  & 12 & 9 & 39.33 & 75.00 \%\\ \hline
\end{tabularx}

\vspace{1cm}
\section{Análisis de Rendimiento}
El análisis del rendimiento académico del primer nivel de TICs muestra una variedad de resultados en las diferentes materias. En cuanto a la materia de Química General, no se reportaron aprobados, lo que sugiere que la materia puede ser un desafío para los estudiantes.

En las materias de Química Analítica, el promedio del curso es de 32.25, y más de la mitad de los estudiantes (56.25\%) han aprobado. Sin embargo, también se observa una gran variabilidad en el rendimiento, con algunos grupos que no han reportado aprobados. Esto puede indicar la necesidad de implementar estrategias de apoyo y recursos adicionales para los estudiantes que lo necesitan.

La materia de Química Organica presenta un promedio del curso de 30.26, y el 47.37\% de los estudiantes han aprobado. De manera similar a Química Analítica, se observa una variedad en el rendimiento, con algunos grupos que no han reportado aprobados. La materia de Bioquimica presenta un promedio del curso de 30.5, y el 43.75\% de los estudiantes han aprobado. Estos resultados sugieren que estas materias pueden ser desafiantes para los estudiantes, especialmente aquellos que no han aprobado.

La materia de Microbiología presenta un promedio del curso de 24.47, y el 52.63\% de los estudiantes han aprobado. Aunque el promedio es bajo, la tasa de aprobados es comparable a otras materias.

Por último, la materia de Fundamentos de la Investigación presenta un promedio del curso de 39.33, y el 75.00\% de los estudiantes han aprobado. Estos resultados sugieren que esta materia puede ser más accesible para los estudiantes, y que otros recursos y estrategias de apoyo pueden ser necesarios para las materias que presentan un rendimiento más bajo.

En general, el análisis del rendimiento académico del primer nivel de TICs sugiere que algunas materias pueden ser desafiantes para los estudiantes, y que es necesario implementar estrategias de apoyo y recursos adicionales para ayudar a aquellos que lo necesitan. Se recomienda también que los docentes ajusten su enfoque y estrategias de enseñanza según las necesidades de los estudiantes y las variaciones en el rendimiento.\\
\vspace{1cm}\\\small
\begin{tabularx}{\textwidth}{|p{2.5cm}|p{2.5cm}|X|X|X|X|}
\hline
\multicolumn{6}{|X|}{\textbf{Nivel: 01 }}\\\hline\textbf{Materia} & \textbf{Docente} & \textbf{Estudiantes} & \textbf{Aprobados} & \textbf{Promedio} & \textbf{\%Supera el Promedio} \\ \hline
Química General & PEÑA ROSAS GLORIA DEL VALLE & 26 & 23 & 33.42 & 57.69 \%\\ \hline
Química General & PEÑA ROSAS GLORIA DEL VALLE & 14 & 0 & 0 & 0.00 \%\\ \hline
Citología e Histología & RUEDA CASTILLO YAJAIRA MARILIN & 26 & 22 & 33.88 & 50.00 \%\\ \hline
Tecnologías de la Información  & RIVERA BONE CARLOS ALEJANDRO & 25 & 22 & 41.16 & 80.00 \%\\ \hline
Comunicación Oral y Escrita & CABALLERO MOREIRA JAIRON ADRIAN & 25 & 23 & 39.76 & 64.00 \%\\ \hline
\end{tabularx}

\vspace{1cm}
\section{Análisis de Rendimiento}
La información presentada revela el rendimiento académico de los estudiantes de primer nivel de TICs, quien han cursado diversas materias. En general, se puede observar que la mayoría de los estudiantes han tenido un buen rendimiento en las materias que han cursado, con promedios que van desde el 33.42 en Química General hasta el 41.16 en Tecnologías de la Información.

En cuanto al número de estudiantes aprobados, se puede ver que la mayoría de las materias tienen un alto porcentaje de aprobados, con los cursos de Anatomía y Fisiología y Tecnologías de la Información que tienen el mayor número de aprobados, con un 60\% y 80\% respectivamente. En cambio, las materias de Química General y Citología e Histología presentan un menor porcentaje de aprobados, con solo el 23\% y el 0\% respectivamente.

Es importante destacar que la materia de Técnicas Básicas de Laboratori tiene un resultado notable, ya que todos los estudiantes que cursaron la materia lograron aprobarla, lo que sugiere que los estudiantes están recibiendo un buen apoyo en esta área. Por otro lado, las materias de Tutorías de Acompañamiento y Citología e Histología no tienen aprobados, lo que puede indicar que necesitan más apoyo y recursos para que los estudiantes puedan tener un buen desempeño en estas materias.

En general, los resultados sugieren que los estudiantes de primer nivel de TICs han tenido un buen rendimiento en la mayoría de las materias, pero también hay áreas donde necesitan más apoyo y recursos para mejorar su desempeño. Es importante que los docentes y directivos realicen análisis más detallados y estrategias para abordar los problemas identificados y fomentar el éxito académico de los estudiantes.\\
\vspace{1cm}\\\small
\begin{tabularx}{\textwidth}{|p{2.5cm}|p{2.5cm}|X|X|X|X|}
\hline
\multicolumn{6}{|X|}{\textbf{Nivel: 05 }}\\\hline\textbf{Materia} & \textbf{Docente} & \textbf{Estudiantes} & \textbf{Aprobados} & \textbf{Promedio} & \textbf{\%Supera el Promedio} \\ \hline
Bioquímica Clínica III & ZÚÑIGA SOSA EVELIN ALEXANDRA & 13 & 0 & 0 & 0.00 \%\\ \hline
Bioquímica Clínica III & ZÚÑIGA SOSA EVELIN ALEXANDRA & 13 & 0 & 0 & 0.00 \%\\ \hline
Inmunohematología & CHILA GARCIA KAREN CAROLINA & 39 & 33 & 30.69 & 61.54 \%\\ \hline
Inmunohematología & CHILA GARCIA KAREN CAROLINA & 13 & 0 & 0 & 0.00 \%\\ \hline
Inmunohematología & CHILA GARCIA KAREN CAROLINA & 16 & 0 & 0 & 0.00 \%\\ \hline
\end{tabularx}

\vspace{1cm}
\section{Análisis de Rendimiento}
El análisis del rendimiento académico de los estudiantes de primer nivel de TICs revela algunos patrones y tendencias importantes. En general, se observa que el promedio del curso es relativamente bajo, con un valor de 0.00 en los cursos de Bioquimica Clínica II, Inmunología Clínica II y algunos de los cursos de Bacteriología Clínica I y Bioquímica Clínica III.

Sin embargo, hay algún curso que destaca por su buen rendimiento. El curso de Bacteriología Clínica I, dado por el docente Castro Demera Dicke Alejandro, tiene un numero significativo de estudiantes aprobados (62.50\%) y un promedio del curso de 28.13. El curso de Virología, dado por el docente Estupiñan Sánchez Alfredo Sandino, también tiene un alto porcentaje de aprobados (63.64\%) y un promedio del curso de 41.77.

En cuanto a los cursos que tienen un rendimiento más mediocre, se puede observar que Bioquimica Clínica III, dado por la docente Zúñiga Sosa Evelin Alexandra, tiene un alta concentración de estudiantes con resultados negativos, lo mismo que el curso Inmunohematología, dado por la docente Chila García Karen Carolina. Esto sugiere la necesidad de revisar y mejorar el diseño y la implementación de estos cursos.

En general, el porcentaje de estudiantes que supera el promedio es relativamente bajo, lo que sugiere que los estudiantes necesitan trabajar para mejorar sus habilidades y conocimientos en los diferentes campos de estudio.

Es importante destacar que todos los cursos tienen asignados el mismo grupo de estudiantes, lo que sugiere que la variabilidad en el rendimiento es debido a factores intrínsecos al curso y no a factores extrínsecos, como la habilidad o disposición individual de los estudiantes. En este sentido, es importante que los docentes revisen y ajusten su estrategias de enseñanza y evaluación para mejorar el rendimiento de sus estudiantes.

En conclusión, el análisis del rendimiento académico de los estudiantes de primer nivel de TICs revela algunos patrones interesantes y desafíos importantes. Es necesario que los docentes y administradores de la institución trabajen juntos para revisar y mejorar los cursos que tienen un rendimiento más mediocre, y para brindar apoyo a los estudiantes que necesitan mejorar sus habilidades y conocimientos.\\
\vspace{1cm}\\\small
\begin{tabularx}{\textwidth}{|p{2.5cm}|p{2.5cm}|X|X|X|X|}
\hline
\multicolumn{6}{|X|}{\textbf{Nivel: 03 }}\\\hline\textbf{Materia} & \textbf{Docente} & \textbf{Estudiantes} & \textbf{Aprobados} & \textbf{Promedio} & \textbf{\%Supera el Promedio} \\ \hline
Bioquimica & ACOSTA GANAN MICHAEL ANDRES & 6 & 0 & 0 & 0.00 \%\\ \hline
Inmunología Clínica I & CHILA GARCIA KAREN CAROLINA & 25 & 16 & 29.44 & 64.00 \%\\ \hline
Inmunología Clínica I & CHILA GARCIA KAREN CAROLINA & 10 & 0 & 0 & 0.00 \%\\ \hline
Inmunología Clínica I & CHILA GARCIA KAREN CAROLINA & 12 & 0 & 0 & 0.00 \%\\ \hline
Hematología I & AGREDA EGAS EYLEN AMANDA & 18 & 12 & 29.39 & 66.67 \%\\ \hline
Hematología I & AGREDA EGAS EYLEN AMANDA & 8 & 0 & 0 & 0.00 \%\\ \hline
\end{tabularx}

\vspace{1cm}
\section{Análisis de Rendimiento}
La información presentada revela un panorama rico y diverso en cuanto al rendimiento académico de los estudiantes de primer nivel de TICs. En lo que respecta a las materias, pueden destacarse algunas tendencias y patrones que son interesantes analizar.

En primer lugar, la materia de Parasitología ha sido la que ha tenido el peor rendimiento, con una gran mayoría de estudiantes que no han aprobado el curso. Las clases impartidas por la docente CHILA GARCIA KAREN CAROLINA han sido las más pobres, con una tasa de aprobación del 0\%. Esto sugiere que la estructura y la organización de las clases pueden ser fundamentales para el éxito del estudiante.

Por otro lado, la materia de Inmunología Clínica I ha sido la que ha tenido el mejor rendimiento, con una tasa de aprobación del 64\% en las diferentes clases impartidas por la docente CHILA GARCIA KAREN CAROLINA. Esto puede indicar que la exposición y la clarificación del material son importantes para el entendimiento y la comprensión de los conceptos.

En cuanto a la materia de Bioquímica, el rendimiento ha sido irregular, con una tasa de aprobación del 0\% en algunas clases y del 50\% en otras. Esto sugiere que la diversidad en la forma en que se impartió la materia puede ser un factor que influye en el rendimiento individual.

La materia de Análisis Instrumental también ha presentado resultados heterogéneos, con tasas de aprobación entre el 44.44\% y el 57.14\%. Esto puede indicar que la complejidad del material y la dificultad para comprender los conceptos son factores que afectan el rendimiento.

La materia de Hematología I ha mostrado un rendimiento similar, con tasas de aprobación entre el 0\% y el 66.67\%. Esto sugiere que la exposición y la práctica de los conceptos son fundamentales para lograr una comprensión profunda.

En cuanto a las materias de Bioquímica Clínica I y Contextos e Interculturalidad, el rendimiento ha sido más homogéneo, con tasas de aprobación importantes (50\% y 61.90\%, respectivamente). Esto puede indicar que la estructura y la organización de las clases, así como la clarificación del material, son fundamentales para el éxito del estudiante.

En conclusión, el análisis de los resultados revela que el rendimiento académico de los estudiantes de primer nivel de TICs es diverso y se encuentra influenciado por factores como la estructura y la organización de las clases, la clarificación del material y la complejidad de los conceptos. Es importante que los docentes reflexionen sobre estos resultados y tomen medidas para mejorar el rendimiento de los estudiantes en cada materia.\\
\vspace{1cm}\\\small
\begin{tabularx}{\textwidth}{|p{2.5cm}|p{2.5cm}|X|X|X|X|}
\hline
\multicolumn{6}{|X|}{\textbf{Nivel: 04 }}\\\hline\textbf{Materia} & \textbf{Docente} & \textbf{Estudiantes} & \textbf{Aprobados} & \textbf{Promedio} & \textbf{\%Supera el Promedio} \\ \hline
Inmunología Clínica I & CHILA GARCIA KAREN CAROLINA & 3 & 0 & 0 & 0.00 \%\\ \hline
Lectura y Escritura Académica & PEÑA ROSAS GLORIA DEL VALLE & 14 & 13 & 34.86 & 64.29 \%\\ \hline
Uroanálisis y Líquidos Biológi & ZÚÑIGA SOSA EVELIN ALEXANDRA & 17 & 13 & 29.71 & 76.47 \%\\ \hline
Etica Personal y Socioambienta & ZAMBRANO DUEÑAS JOSE MARIA & 9 & 9 & 44.44 & 55.56 \%\\ \hline
\end{tabularx}

\vspace{1cm}
\section{Análisis de Rendimiento}
El análisis del rendimiento académico de los estudiantes de primer nivel de TICs revela resultados heterogéneos en términos de número de aprobados y promedios del curso en cada materia. A continuación, se presentan los resultados más destacados:

En la materia de Inmunología Clínica I, con un total de 3 estudiantes matriculados, no hubo aprobados, resultado que se refleja en un promedio del curso de 0.00 y un porcentaje de estudiantes que supera el promedio de 0.

Por otro lado, la materia de Lectura y Escritura Académica presenta un rendimiento notable, con 14 estudiantes matriculados, 13 aprobados y un promedio del curso de 34.86. El porcentaje de estudiantes que supera el promedio es de 64.29, lo que sugiere una buena comprensión y habilidad en las competencias de lectura y escritura.

En cuanto a la materia de Uroanálisis y Líquidos Biológi, se observa un desempeño dispuesto, con 17, 11 y 6 estudiantes matriculados respectivamente. Sin embargo, no hubo aprobados en dos de las tres secciones, lo que se refleja en promedios del curso y porcentajes de estudiantes que superan el promedio muy bajos. Se sugiere que se requiere un mayor esfuerzo para mejorar el rendimiento en esta materia.

En la materia de Hematología II, se vio un desempeño heterogéneo, con 18, 12 y 6 estudiantes matriculados respectivamente. Los resultados son mixtos, con 16 aprobados en la sección mayor y no aprobados en las otras dos, lo que se refleja en promedios del curso y porcentajes de estudiantes que superan el promedio de 44.44 y 0 respectivamente.

La materia de Bioquimica Clínica II presentó un rendimiento aceptable, con 15 estudiantes matriculados, 13 aprobados y un promedio del curso de 34.93. El porcentaje de estudiantes que supera el promedio es de 53.33, indicando una buena comprensión de los conceptos bioquímicos.

La materia de Inmunología Clínica II presentó un rendimiento similar al de la materia de Hematología II, con 14 estudiantes matriculados, 12 aprobados y un promedio del curso de 35.07. El porcentaje de estudiantes que supera el promedio es de 42.86, lo que sugiere que se requiere un mayor esfuerzo para mejorar el rendimiento en esta materia.

La materia de Citología Clínica no presentó aprobados, lo que se refleja en un promedio del curso y un porcentaje de estudiantes que supera el promedio de 0.00.

Finalmente, en la materia de Tutorías de Acompañamiento y Etica Personal y Socioambiental no hubo aprobados, resultado que se refleja en promedios del curso y porcentajes de estudiantes que superan el promedio de 0.00.

En conclusion, los resultados del análisis del rendimiento académico de los estudiantes de primer nivel de TICs revelan un desempeño heterogéneo, con materias que presentan buenos resultados y otras que requieren un mayor esfuerzo. Es importante que se identifiquen las causas detrás de estos resultados y se implementen estrategias para mejorar la calidad de la educación.\\
\vspace{1cm}\\\small
\begin{tabularx}{\textwidth}{|p{2.5cm}|p{2.5cm}|X|X|X|X|}
\hline
\multicolumn{6}{|X|}{\textbf{Nivel: 06 }}\\\hline\textbf{Materia} & \textbf{Docente} & \textbf{Estudiantes} & \textbf{Aprobados} & \textbf{Promedio} & \textbf{\%Supera el Promedio} \\ \hline
Control de Calidad I & AGREDA EGAS EYLEN AMANDA & 17 & 16 & 34.24 & 58.82 \%\\ \hline
Diseño y Evaluación de Proyect & ESTUPIÑAN SANCHEZ ALFREDO SANDINO & 30 & 29 & 38.27 & 56.67 \%\\ \hline
Patología Clinica & CALDERON RUIZ PAOLA ALEXANDRA & 27 & 27 & 44.96 & 62.96 \%\\ \hline
Tutorías de Acompañamiento & ACOSTA GANAN MICHAEL ANDRES & 28 & 0 & 0 & 0.00 \%\\ \hline
\end{tabularx}

\vspace{1cm}
\section{Análisis de Rendimiento}
Analizando los datos de rendimiento académico presentados, se puede observar que el curso de primer nivel de TICs ha tenido un desempeño generalizado en las diversas materias que se ofrecen. Sin embargo, existen algunas tendencias y patrones que se pueden destacar.

En cuanto a la materia de Bacteriología Clínica II, se ha observado un rendimiento deficitario en sus tres aplicaciones, con un número de aprobados bajo y un promedio significativamente bajo. Esto sugiere que la materia necesitaría un enfoque más intensivo para mejorar el desempeño de los estudiantes.

En contraste, algunas materias han destacado por su alto rendimiento. Por ejemplo, la materia de Patología Clínica ha tenido un porcentaje de aprobados cercano al 100\%, con un promedio muy alto. Esto sugiere que la materia está siendo enseñada de manera efectiva y que los estudiantes están logrando comprender los conceptos.

La materia de Diseño y Evaluación de Proyectos también ha mostrado un buen rendimiento, con un porcentaje de aprobados alto y un promedio moderadamente alto. Esto sugiere que los estudiantes están desarrollando habilidades valiosas en áreas como la creatividad y la resolución de problemas.

En cuanto a la materia de Bioestadística I, se ha observado un rendimiento moderado, con un número de aprobados relativamente alto pero un promedio bajo. Esto sugiere que los estudiantes necesitan trabajar más para desarrollar habilidades estadísticas efectivas.

Por otro lado, las materias de Genética y Biología Molecular y Tutorías de Acompañamiento han tenido un rendimiento discreto, con un número de aprobados alto pero un promedio moderadamente bajo. Esto sugiere que los estudiantes necesitan trabajar más para desarrollar habilidades en áreas específicas.

En cuanto al instructor, se puede observar que algunas materias han tenido un rendimiento más alto que otras, lo que sugiere que el instructor puede tener un efecto significativo en el rendimiento de los estudiantes. Sin embargo, no se puede determinar conclusivamente si el instructor es el factor determinante en el rendimiento, ya que otros factores pueden influir, como la preparación previa de los estudiantes y la cantidad de tiempo que dedican a estudiar.

En general, el rendimiento del curso ha sido satisfactorio en algunas materias, pero hay áreas que necesitan mejorar. Se recomienda implementar estrategias de apoyo adicional para estudiantes que strugglen y proporcionar retroalimentación constructiva a los instructores para ayudar a mejorar el rendimiento del curso.\\
\vspace{1cm}\\\small
\begin{tabularx}{\textwidth}{|p{2.5cm}|p{2.5cm}|X|X|X|X|}
\hline
\multicolumn{6}{|X|}{\textbf{Nivel: 07 }}\\\hline\textbf{Materia} & \textbf{Docente} & \textbf{Estudiantes} & \textbf{Aprobados} & \textbf{Promedio} & \textbf{\%Supera el Promedio} \\ \hline
Control de Calidad II & ZÚÑIGA SOSA EVELIN ALEXANDRA & 19 & 16 & 33.32 & 31.58 \%\\ \hline
Micología Médica & VIZCAINO ORDOÑEZ LAURA DEL ROCIO  & 22 & 20 & 33.73 & 45.45 \%\\ \hline
Micología Médica & VIZCAINO ORDOÑEZ LAURA DEL ROCIO  & 11 & 0 & 0 & 0.00 \%\\ \hline
Micología Médica & VIZCAINO ORDOÑEZ LAURA DEL ROCIO  & 11 & 0 & 0 & 0.00 \%\\ \hline
Tutorías de Acompañamiento & RUEDA CASTILLO YAJAIRA MARILIN & 28 & 0 & 0 & 0.00 \%\\ \hline
\end{tabularx}

\vspace{1cm}
\section{Análisis de Rendimiento}
El análisis del rendimiento académico de los estudiantes de primer nivel de TICs revela un panorama diverso en cada materia. En general, se observa que la mayoría de las materias presentan un promedio considerable, lo que indica que los estudiantes tienen una buena comprensión de los conceptos y habilidades envasadas.

En particular, la matería de Micología Médica, impartida por la docente Vizcaíno Ordóñez Laura del Rocío, presenta un promedio notable, siendo el 33.73 y 45.45 en dos de las oportunidades que se presentan. Esto sugiere que los estudiantes han logrado una buena comprensión de los conceptos relacionados con la micología médica.

Por otro lado, las materias de Bioestadística II y Prácticas de Servicio Comunitario presentan un promedio distinto. La matería de Bioestadística II, impartida por la docente Mera Carranza Josellyn Genesis, tiene un promedio de 40.69 y un porcentaje de aprobados de 56.25, lo que indica que los estudiantes han logrado una buena comprensión de los conceptos estadísticos y han sido capaces de aplicarlos.

La materia de Prácticas de Servicio Comunitario, impartida por la docente Zúñiga Sosa Evelin Alexandra, tiene un promedio de 45.58 y un porcentaje de aprobados de 58.33, lo que sugiere que los estudiantes han logrado desarrollar habilidades prácticas y han sido capaces de aplicarlos en situaciones reales.

Es importante destacar que algunas materias presentan un promedio y un porcentaje de aprobados más bajos. En particular, las materias de Micología Médica, Enfermedades Tropicales y Tutorías de Acompañamiento presentan un promedio y un porcentaje de aprobados muy bajos, lo que sugiere que los estudiantes pueden estar requiriendo más apoyo y orientación para completar con éxito estas materias.

En conclusión, el análisis del rendimiento académico de los estudiantes de primer nivel de TICs revela un panorama diverso en cada materia. Aunque algunas materias presentan un promedio y un porcentaje de aprobados más altos, otras necesitan más apoyo y orientación. Es importante que los docentes y las instituciones educativas tomen medidas para apoyar a los estudiantes que están luchando y mejorar la calidad de la educación.\\
\vspace{1cm}\\\small
\begin{tabularx}{\textwidth}{|p{2.5cm}|p{2.5cm}|X|X|X|X|}
\hline
\multicolumn{6}{|X|}{\textbf{Nivel: 08 }}\\\hline\textbf{Materia} & \textbf{Docente} & \textbf{Estudiantes} & \textbf{Aprobados} & \textbf{Promedio} & \textbf{\%Supera el Promedio} \\ \hline
Administración de Laboratorio & SALAZAR DONOSO ELIAS HUMBERTO & 15 & 15 & 44.87 & 66.67 \%\\ \hline
Toxicología & PEÑA ROSAS GLORIA DEL VALLE & 13 & 13 & 39.38 & 53.85 \%\\ \hline
Análisis y Validación de Datos & RUEDA CASTILLO YAJAIRA MARILIN & 14 & 12 & 32.64 & 57.14 \%\\ \hline
\end{tabularx}

\vspace{1cm}
\section{Análisis de Rendimiento}
El análisis de los resultados del rendimiento académico de los estudiantes del primer nivel de TICs muestra una variedad de tendencias y patrones. En general, las materias de Enfermedades Tropicales y Análisis y Validación de Datos registran resultados más bajos, con promedios globales de 34.23 y 32.64 respectivamente. El curso de Enfermedades Tropicales, impartido por Castro Demera Dicke Alejandro, tuvo un resultado particularmente decepcionante, con solo un 0\% de aprobativos y un promedio de 0.00.

Por otro lado, las materias de Toxicología y Administración de Laboratorio han mostrado resultados más satisfactorios. La materia de Toxicología, impartida por Peña Rosas Gloria del Valle, tiene un promedio de 39.38 y un 53.85\% de estudiantes que superaron el promedio. De igual manera, el curso de Administración de Laboratorio, impartido por Salazar Donoso Elias Humberto, tiene un promedio de 44.87 y un 66.67\% de estudiantes que aprobaron.

Además, es importante destacar que el porcentaje de estudiantes que superan el promedio es variable según la materia. En Toxicología y Administración de Laboratorio, el 53.85\% y 66.67\% de estudiantes respectivamente, superan el promedio, mientras que en las materias de Enfermedades Tropicales y Análisis y Validación de Datos, menos del 30\% de estudiantes superan el promedio.

En resumen, el análisis de los resultados del rendimiento académico muestra que las materias de Toxicología y Administración de Laboratorio han sido más exitosas, mientras que las materias de Enfermedades Tropicales y Análisis y Validación de Datos han registro resultados más desalentadores. Esta información puede ser utilizada para identificar áreas de mejora y diseñar estrategias para apoyar a los estudiantes y mejorar la calidad de la educación en el primer nivel de TICs.\\
\vspace{1cm}\\\begin{tabularx}{\textwidth}{|X|X|X|}
    \hline
    \textbf{ELABORADO POR:} & \textbf{REVISADO POR:} & \textbf{APROBADO POR:} \\ \hline
    Firma: & Firma: & Firma:\\
    &&\\
    &&\\
    &&\\ \hline
    \textbf{Nombre: Homero Velasteguí} & \textbf{Nombre: Manuel Nevarez} & \textbf{Nombre: Pablo Pico Valencia PhD.} \\ \hline
    \textbf{Cargo: Coordinador Carrera} & \textbf{Cargo: Consejo de Escuela} & \textbf{Cargo: Director Académico} \\ \hline
    \textbf{Fecha: 9/3/2024} & \textbf{Fecha: 9/3/2024} & \textbf{Fecha: 9/3/2024} \\ \hline
    \end{tabularx}
