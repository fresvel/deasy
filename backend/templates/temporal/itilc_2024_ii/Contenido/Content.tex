\small
\begin{tabularx}{\textwidth}{|p{2.5cm}|p{2.5cm}|X|X|X|X|}
\hline
\multicolumn{6}{|X|}{\textbf{Nivel: 02 }}\\\hline\textbf{Materia} & \textbf{Docente} & \textbf{Estudiantes} & \textbf{Aprobados} & \textbf{Promedio} & \textbf{\%Supera el Promedio} \\ \hline
Química General & PEÑA ROSAS GLORIA DEL VALLE & 12 & 0 & 0 & 0.00 \%\\ \hline
Química Analítica & PEÑA ROSAS GLORIA DEL VALLE & 16 & 13 & 32.25 & 56.25 \%\\ \hline
Química Analítica & PEÑA ROSAS GLORIA DEL VALLE & 9 & 0 & 0 & 0.00 \%\\ \hline
Química Analítica & PEÑA ROSAS GLORIA DEL VALLE & 7 & 0 & 0 & 0.00 \%\\ \hline
Quimica Organica & ACOSTA GANAN MICHAEL ANDRES & 19 & 13 & 30.26 & 47.37 \%\\ \hline
Quimica Organica & ACOSTA GANAN MICHAEL ANDRES & 11 & 0 & 0 & 0.00 \%\\ \hline
Quimica Organica & ACOSTA GANAN MICHAEL ANDRES & 8 & 0 & 0 & 0.00 \%\\ \hline
Bioquimica & ACOSTA GANAN MICHAEL ANDRES & 16 & 8 & 30.5 & 43.75 \%\\ \hline
Bioquimica & ACOSTA GANAN MICHAEL ANDRES & 10 & 0 & 0 & 0.00 \%\\ \hline
Microbiología & CASTRO DEMERA DICKE ALEJANDRO & 19 & 8 & 24.47 & 52.63 \%\\ \hline
Microbiología & CASTRO DEMERA DICKE ALEJANDRO & 13 & 0 & 0 & 0.00 \%\\ \hline
Microbiología & CASTRO DEMERA DICKE ALEJANDRO & 6 & 0 & 0 & 0.00 \%\\ \hline
Tutorías de Acompañamiento & PEÑA ROSAS GLORIA DEL VALLE & 18 & 0 & 0 & 0.00 \%\\ \hline
Fundamentos de la Investigació & SUAREZ LEZCANO JOSE  & 12 & 9 & 39.33 & 75.00 \%\\ \hline
\end{tabularx}

\vspace{1cm}
\section{Análisis de Rendimiento}
El análisis del rendimiento académico de los estudiantes de primer nivel de TICs revela resultados variados dependiendo de la materia y el profesor. En general, se observa que la mayoría de las materias tienen una cantidad significativa de estudiantes que no aprobaron el curso, lo que sugiere una gran cantidad de estudiante que necesitan recibir apoyo y recursos adicionales para mejorar su rendimiento.

En Química Analítica, se observó que solo un grupo de 16 estudiantes aprobaron el curso, con un promedio de 32.25, lo que indica que la mayoría de los estudiantes no lograron aprobar. Por otro lado, en Química Organica, se encontró un promedio de 30.26 con 13 aprobados, lo que sugiere un resultado moderado.

En Bioquimica y Microbiología, se observó un resultado similar, con promedios de 30.5 y 24.47, respectivamente, y un número de aprobados relativamente bajo. Sin embargo, en Fundamentos de la Investigación, se encontró un promedio de 39.33 con 9 aprobados, lo que sugiere un resultado positivo.

En cuanto al porcentaje de estudiantes que supera el promedio, se puede observar que en el 52.63\% de los casos, el promedio es superado, lo que indica que una gran cantidad de estudiantes tienen un rendimiento académico satisfactorio.

En resumen, los resultados del análisis del rendimiento académico sugieren que las materias de Química y Bioquimica Microbiología están presentando algún tipo de problema, ya que los estudiantes no están logrando aprobar en una cantidad significativa. Por otro lado, la materia de Fundamentos de la Investigación está mostrando un resultado más positivo, lo que sugiere que el enfoque y los recursos utilizados en ese curso están siendo efectivos. Es importante que se realice un análisis más profundo de los resultados y se identifiquen las causas del bajo rendimiento en algunas materias, con el fin de implementar estrategias de apoyo y recursos adicionales para ayudar a los estudiantes a mejorar su rendimiento.\\
\vspace{1cm}\\\small
\begin{tabularx}{\textwidth}{|p{2.5cm}|p{2.5cm}|X|X|X|X|}
\hline
\multicolumn{6}{|X|}{\textbf{Nivel: 01 }}\\\hline\textbf{Materia} & \textbf{Docente} & \textbf{Estudiantes} & \textbf{Aprobados} & \textbf{Promedio} & \textbf{\%Supera el Promedio} \\ \hline
Química General & PEÑA ROSAS GLORIA DEL VALLE & 26 & 23 & 33.42 & 57.69 \%\\ \hline
Química General & PEÑA ROSAS GLORIA DEL VALLE & 14 & 0 & 0 & 0.00 \%\\ \hline
Citología e Histología & RUEDA CASTILLO YAJAIRA MARILIN & 26 & 22 & 33.88 & 50.00 \%\\ \hline
Citología e Histología & RUEDA CASTILLO YAJAIRA MARILIN & 15 & 0 & 0 & 0.00 \%\\ \hline
Citología e Histología & RUEDA CASTILLO YAJAIRA MARILIN & 11 & 0 & 0 & 0.00 \%\\ \hline
Anatomía y Fisiología & ESTUPIÑAN SANCHEZ ALFREDO SANDINO & 25 & 24 & 40.32 & 60.00 \%\\ \hline
Anatomía y Fisiología & ESTUPIÑAN SANCHEZ ALFREDO SANDINO & 14 & 0 & 0 & 0.00 \%\\ \hline
Anatomía y Fisiología & ESTUPIÑAN SANCHEZ ALFREDO SANDINO & 11 & 0 & 0 & 0.00 \%\\ \hline
Técnicas Básicas de Laboratori & HIDALGO TAPIA COSME ENRIQUE & 14 & 14 & 34 & 42.86 \%\\ \hline
Técnicas Básicas de Laboratori & HIDALGO TAPIA COSME ENRIQUE & 11 & 10 & 32.91 & 72.73 \%\\ \hline
Tutorías de Acompañamiento & PEÑA ROSAS GLORIA DEL VALLE & 26 & 0 & 0 & 0.00 \%\\ \hline
Tecnologías de la Información  & RIVERA BONE CARLOS ALEJANDRO & 25 & 22 & 41.16 & 80.00 \%\\ \hline
Comunicación Oral y Escrita & CABALLERO MOREIRA JAIRON ADRIAN & 25 & 23 & 39.76 & 64.00 \%\\ \hline
\end{tabularx}

\vspace{1cm}
\section{Análisis de Rendimiento}
El análisis del rendimiento académico de los estudiantes de primer nivel de TICs presenta resultadosmixedos. En general, se observa que la mayoría de las materias tienen un porcentaje alto de aprobados, lo que sugiere un buen nivel de comprensión y dominio de los contenidos por parte de los estudiantes. Sin embargo, también se encuentran algunas materias con un porcentaje bajo de aprobados, lo que puede indicar problemas de comprensión o retención de los aprendizajes.

Las materias de Química General, Citología e Histología y Anatomía y Fisiología presentan resultados mixtos. En Química General, se observa que solo el 44,23\% de los estudiantes aprobó la materia, lo que podría ser debido a la complejidad de los conceptos y la importancia de la comprensión teórica y practica. En Citología e Histología y Anatomía y Fisiología, se encuentra un porcentaje bajo de aprobados, lo que puede indicar necesidad de apoyo adicional para los estudiantes.

Por otro lado, las materias de Técnicas Básicas de Laboratorio y Tecnologías de la Información presentan resultados más positivos. En Técnicas Básicas de Laboratorio, se observa que el 71,43\% de los estudiantes aprobó la materia, lo que sugiere que los estudiantes tienen una buena comprensión de los conceptos y habilidades laboratorios. En Tecnologías de la Información, se encuentra un porcentaje alto de aprobados, con un 88\% de los estudiantes aprobando la materia, lo que sugiere que los estudiantes tienen una buena comprensión de los conceptos y habilidades relacionados con el uso de las tecnologías de la información.

En cuanto al promedio del curso, se observa que las materias de Química General, Citología e Histología y Anatomía y Fisiología tienen promedios menores a 40, mientras que las materias de Técnicas Básicas de Laboratorio y Tecnologías de la Información tienen promedios altos, superiores a 40. Esto sugiere que los estudiantes tienen una buena comprensión de los conceptos y habilidades en las materias que involucran la aplicación práctica y el uso de tecnologías.

En resumen, el análisis del rendimiento académico de los estudiantes de primer nivel de TICs presenta resultados mixtos. Aunque algunas materias presentan un porcentaje de aprobados alto, otras presentan resultados más negativos. Se deben tomar medidas para apoyar a los estudiantes que enfrentan dificultades en algunas materias y para desarrollar estrategias para mejorar la comprensión y retención de los aprendizajes.\\
\vspace{1cm}\\\small
\begin{tabularx}{\textwidth}{|p{2.5cm}|p{2.5cm}|X|X|X|X|}
\hline
\multicolumn{6}{|X|}{\textbf{Nivel: 05 }}\\\hline\textbf{Materia} & \textbf{Docente} & \textbf{Estudiantes} & \textbf{Aprobados} & \textbf{Promedio} & \textbf{\%Supera el Promedio} \\ \hline
Bioquimica Clínica II & HIDALGO TAPIA COSME ENRIQUE & 2 & 0 & 0 & 0.00 \%\\ \hline
Inmunología Clínica II & RUEDA CASTILLO YAJAIRA MARILIN & 3 & 0 & 0 & 0.00 \%\\ \hline
Bacteriología Clínica I & CASTRO DEMERA DICKE ALEJANDRO & 32 & 20 & 28.13 & 62.50 \%\\ \hline
Bacteriología Clínica I & CASTRO DEMERA DICKE ALEJANDRO & 14 & 0 & 0 & 0.00 \%\\ \hline
Bacteriología Clínica I & CASTRO DEMERA DICKE ALEJANDRO & 18 & 0 & 0 & 0.00 \%\\ \hline
Epidemiología y Realidad Nacio & ACOSTA GANAN MICHAEL ANDRES & 26 & 17 & 30.88 & 53.85 \%\\ \hline
Virología & ESTUPIÑAN SANCHEZ ALFREDO SANDINO & 22 & 22 & 41.77 & 63.64 \%\\ \hline
Bioquímica Clínica III & ZÚÑIGA SOSA EVELIN ALEXANDRA & 26 & 23 & 32.85 & 57.69 \%\\ \hline
Bioquímica Clínica III & ZÚÑIGA SOSA EVELIN ALEXANDRA & 13 & 0 & 0 & 0.00 \%\\ \hline
Bioquímica Clínica III & ZÚÑIGA SOSA EVELIN ALEXANDRA & 13 & 0 & 0 & 0.00 \%\\ \hline
Inmunohematología & CHILA GARCIA KAREN CAROLINA & 39 & 33 & 30.69 & 61.54 \%\\ \hline
Inmunohematología & CHILA GARCIA KAREN CAROLINA & 13 & 0 & 0 & 0.00 \%\\ \hline
Inmunohematología & CHILA GARCIA KAREN CAROLINA & 16 & 0 & 0 & 0.00 \%\\ \hline
Citología Clínica & ESTUPIÑAN SANCHEZ ALFREDO SANDINO & 25 & 25 & 43.16 & 48.00 \%\\ \hline
Citología Clínica & ESTUPIÑAN SANCHEZ ALFREDO SANDINO & 12 & 0 & 0 & 0.00 \%\\ \hline
Tutorías de Acompañamiento & ACOSTA GANAN MICHAEL ANDRES & 32 & 0 & 0 & 0.00 \%\\ \hline
Inmunohematología & CHILA GARCIA KAREN CAROLINA & 10 & 0 & 0 & 0.00 \%\\ \hline
\end{tabularx}

\vspace{1cm}
\section{Análisis de Rendimiento}
El análisis del rendimiento académico de los estudiantes de primer nivel de TICs, presentado en la tabla, permite identificar tendencias y patrones en el desempeño de los estudiantes en diferentes materias. En general, se observa que las materias que presentan un mayor número de aprobados son Bacteriología Clínica I, Epidemiología y Realidad Nacional, Virología, Bioquímica Clínica III y Inmunohematología, con más del 50\% de aprobados. Por otro lado, materiales como Bioquímica Clínica II, Inmunología Clínica II y Citología Clínica presentan un menor número de aprobados, con fewer than 10\% de aprobados.

En cuanto al promedio del curso, se observa que las materias que presentan un mayor promedio son Virología, Bioquímica Clínica III, y Inmunohematología, con promedios superiores a 30. Este resultado indica que los estudiantes han demostrado un buen desempeño en estas materias, lo que sugiere una buena comprensión de los conceptos y principios relacionados con la especialidad. Por otro lado, materiales como Bioquímica Clínica II y Citología Clínica presentan promedios más bajos, lo que puede indicar que los estudiantes necesitan mejoras en la comprensión de los conceptos básicos y la aplicación de las habilidades y técnicas.

El porcentaje de estudiantes que supera el promedio en el curso es otro indicador importante del rendimiento académico. En general, se observa que alrededor del 50-70\% de los estudiantes supera el promedio en las materias que presentan promedios más altos. Sin embargo, en materiales como Citología Clínica y Bioquímica Clínica II, solo un pequeño porcentaje de estudiantes supera el promedio, lo que sugiere que muchos estudiantes necesitan mejorar su desempeño en estas materias.

En resumen, el análisis del rendimiento académico de los estudiantes de primer nivel de TICs indica que los estudiantes tienen una buena comprensión de los conceptos y principios relacionados con la especialidad en materias como Virología, Bioquímica Clínica III, y Inmunohematología. Sin embargo, en materiales como Bioquímica Clínica II, Inmunología Clínica II, y Citología Clínica, los estudiantes necesitan mejorar su comprensión de los conceptos básicos y la aplicación de las habilidades y técnicas. Se recomienda que los docentes brinden apoyo adicional a los estudiantes en estas materias y que se implementen estrategias para mejorar la comprensión de los conceptos y principios relacionados con la especialidad.\\
\vspace{1cm}\\\small
\begin{tabularx}{\textwidth}{|p{2.5cm}|p{2.5cm}|X|X|X|X|}
\hline
\multicolumn{6}{|X|}{\textbf{Nivel: 03 }}\\\hline\textbf{Materia} & \textbf{Docente} & \textbf{Estudiantes} & \textbf{Aprobados} & \textbf{Promedio} & \textbf{\%Supera el Promedio} \\ \hline
Bioquimica & ACOSTA GANAN MICHAEL ANDRES & 6 & 0 & 0 & 0.00 \%\\ \hline
Parasitología & CHILA GARCIA KAREN CAROLINA & 19 & 12 & 28.53 & 63.16 \%\\ \hline
Parasitología & CHILA GARCIA KAREN CAROLINA & 9 & 0 & 0 & 0.00 \%\\ \hline
Parasitología & CHILA GARCIA KAREN CAROLINA & 10 & 0 & 0 & 0.00 \%\\ \hline
Análisis Instrumental & ACOSTA GANAN MICHAEL ANDRES & 14 & 8 & 27.79 & 57.14 \%\\ \hline
Análisis Instrumental & ACOSTA GANAN MICHAEL ANDRES & 9 & 8 & 32.78 & 44.44 \%\\ \hline
Inmunología Clínica I & CHILA GARCIA KAREN CAROLINA & 25 & 16 & 29.44 & 64.00 \%\\ \hline
Inmunología Clínica I & CHILA GARCIA KAREN CAROLINA & 10 & 0 & 0 & 0.00 \%\\ \hline
Inmunología Clínica I & CHILA GARCIA KAREN CAROLINA & 12 & 0 & 0 & 0.00 \%\\ \hline
Hematología I & AGREDA EGAS EYLEN AMANDA & 18 & 12 & 29.39 & 66.67 \%\\ \hline
Hematología I & AGREDA EGAS EYLEN AMANDA & 8 & 0 & 0 & 0.00 \%\\ \hline
Hematología I & AGREDA EGAS EYLEN AMANDA & 7 & 0 & 0 & 0.00 \%\\ \hline
Hematología I & AGREDA EGAS EYLEN AMANDA & 3 & 0 & 0 & 0.00 \%\\ \hline
Bioquímica Clínica I & HIDALGO TAPIA COSME ENRIQUE & 20 & 18 & 32.15 & 50.00 \%\\ \hline
Bioquímica Clínica I & HIDALGO TAPIA COSME ENRIQUE & 17 & 0 & 0 & 0.00 \%\\ \hline
Bioquímica Clínica I & HIDALGO TAPIA COSME ENRIQUE & 3 & 0 & 0 & 0.00 \%\\ \hline
Tutorías de Acompañamiento & CHILA GARCIA KAREN CAROLINA & 23 & 0 & 0 & 0.00 \%\\ \hline
Contextos e Interculturalidad & BAUTISTA MEJIA KATIUSKA MARIA & 21 & 18 & 38.1 & 61.90 \%\\ \hline
Jesucristo y la Persona de Hoy & QUINTERO ROSALES FRANCISCO JHONNY & 12 & 7 & 30.5 & 41.67 \%\\ \hline
\end{tabularx}

\vspace{1cm}
\section{Análisis de Rendimiento}
El análisis del rendimiento académico de los estudiantes de primer nivel de TICs revela resultados diversos en las diferentes materias. En cuanto a la materia de Parasitología, se observa que las clases con la docente CHILA GARCIA KAREN CAROLINA no han obtenido buenos resultados, ya que en las tres clases se registró un número elevado de estudiantes sin aprobación (0 aprobados en cada clase). En contraste, la materia de Inmunología Clínica I con la misma docente presenta mejores resultados, ya que en una de las clases se obtuvieron 16 aprobados, lo que representa un 64\% de aprobación.

En cuanto a la materia de Hematología I, se observa que la docente AGREDA EGAS EYLEN AMANDA obtuvo mejores resultados en una de las clases, con 12 aprobados, lo que representa un 66.67\% de aprobación. Sin embargo, en las demás clases se registró un número elevado de estudiantes sin aprobación.

En la materia de Bioquímica, se observa que la docente HIDALGO TAPIA COSME ENRIQUE obtuvo buenos resultados en una de las clases, con 18 aprobados, lo que representa un 50\% de aprobación. Sin embargo, en las demás clases se registró un número elevado de estudiantes sin aprobación.

En cuanto al promedio del curso, se observa que las materias de Parasitología y Bioquímica no presentan un promedio significativo, ya que en la mayoría de las clases se registró un número elevado de estudiantes sin aprobación. Por otro lado, las materias de Inmunología Clínica I, Hematología I y Bioquímica Clínica I presentan promedios más altos, con un rango de 27.79 a 38.1.

En resumen, el análisis del rendimiento académico de los estudiantes de primer nivel de TICs revela resultados dispares en las diferentes materias, con algunas clases obteniendo buenos resultados y otras no. Es importante que las docentes tengan en cuenta estos resultados y ajusten su estrategia de enseñanza para mejorar el rendimiento de los estudiantes.\\
\vspace{1cm}\\\small
\begin{tabularx}{\textwidth}{|p{2.5cm}|p{2.5cm}|X|X|X|X|}
\hline
\multicolumn{6}{|X|}{\textbf{Nivel: 04 }}\\\hline\textbf{Materia} & \textbf{Docente} & \textbf{Estudiantes} & \textbf{Aprobados} & \textbf{Promedio} & \textbf{\%Supera el Promedio} \\ \hline
Inmunología Clínica I & CHILA GARCIA KAREN CAROLINA & 3 & 0 & 0 & 0.00 \%\\ \hline
Lectura y Escritura Académica & PEÑA ROSAS GLORIA DEL VALLE & 14 & 13 & 34.86 & 64.29 \%\\ \hline
Uroanálisis y Líquidos Biológi & ZÚÑIGA SOSA EVELIN ALEXANDRA & 17 & 13 & 29.71 & 76.47 \%\\ \hline
Uroanálisis y Líquidos Biológi & ZÚÑIGA SOSA EVELIN ALEXANDRA & 11 & 0 & 0 & 0.00 \%\\ \hline
Uroanálisis y Líquidos Biológi & ZÚÑIGA SOSA EVELIN ALEXANDRA & 6 & 0 & 0 & 0.00 \%\\ \hline
Hematología II & RUEDA CASTILLO YAJAIRA MARILIN & 18 & 16 & 34.44 & 44.44 \%\\ \hline
Hematología II & RUEDA CASTILLO YAJAIRA MARILIN & 12 & 0 & 0 & 0.00 \%\\ \hline
Hematología II & RUEDA CASTILLO YAJAIRA MARILIN & 6 & 0 & 0 & 0.00 \%\\ \hline
Bioquimica Clínica II & HIDALGO TAPIA COSME ENRIQUE & 15 & 13 & 34.93 & 53.33 \%\\ \hline
Bioquimica Clínica II & HIDALGO TAPIA COSME ENRIQUE & 13 & 0 & 0 & 0.00 \%\\ \hline
Inmunología Clínica II & RUEDA CASTILLO YAJAIRA MARILIN & 14 & 12 & 35.07 & 42.86 \%\\ \hline
Inmunología Clínica II & RUEDA CASTILLO YAJAIRA MARILIN & 11 & 0 & 0 & 0.00 \%\\ \hline
Citología Clínica & ESTUPIÑAN SANCHEZ ALFREDO SANDINO & 13 & 0 & 0 & 0.00 \%\\ \hline
Tutorías de Acompañamiento & CHILA GARCIA KAREN CAROLINA & 20 & 0 & 0 & 0.00 \%\\ \hline
Etica Personal y Socioambienta & ZAMBRANO DUEÑAS JOSE MARIA & 9 & 9 & 44.44 & 55.56 \%\\ \hline
\end{tabularx}

\vspace{1cm}
\section{Análisis de Rendimiento}
La información presentada muestra el rendimiento académico de los estudiantes del primer nivel de TICs en diferentes materias. A continuación, se realiza un análisis de los resultados para obtener una visión general del desempeño de los estudiantes.

En general, se observa que los estudiantes de primer nivel de TICs han presentado un rendimiento heterogéneo en las diferentes materias. Los resultados más destacados se han obtenido en las materias de Uroanálisis y Líquidos Biológi, con un promedio del curso de 29.71 y un porcentaje de superación del promedio del 76.47\%. Esta materia ha sido una de las más exitosas para los estudiantes, lo que sugiere que los docentes involucrados, especialmente la profesora ZÚÑIGA SOSA EVELIN ALEXANDRA, han implementado estrategias efectivas para el aprendizaje.

Por otro lado, las materias de Inmunología Clínica I y Tutorías de Acompañamiento han presentado el rendimiento más bajo, con un promedio del curso de 0.00 y no hay aprobados respectivamente. Esto sugiere que necesitan ser replanteadas o reevaluadas para mejorar el rendimiento de los estudiantes.

En las materias de Uroanálisis y Líquidos Biológi y Bioquimica Clínica II, se observa un buen desempeño, con promedios del curso entre 29.71 y 34.93 y un porcentaje de superación del promedio entre 44.44 y 64.29\%. Sin embargo, hay un número importante de estudiante no aprobados en estas materias, lo que sugiere que hay un gap en el aprendizaje que necesita ser abordado.

En contraste, las materias de Hematología II y Inmunología Clínica II han presentado un rendimiento más moderado, con promedios del curso entre 34.44 y 35.07 y un porcentaje de superación del promedio entre 42.86 y 55.56\%. Aunque no son resultados notables, no hay una gran cantidad de estudiantes no aprobados en estas materias.

Por último, las materias de Citología Clínica y Etica Personal y Socioambiental han presentado resultados desiguales, con un número importante de estudiantes no aprobados en Citología Clínica y un buen desempeño en Etica Personal y Socioambiental.

En resumen, los resultados del curso han sido variados, con algunas materias presentando buenos resultados y otras necesitando ser reevaluados. Es importante que los docentes y los administradores involucrados en el curso revisen y replanteen las estrategias de enseñanza para mejorar el rendimiento de los estudiantes y asegurar que todos tengan oportunidades de aprendizaje y éxito.\\
\vspace{1cm}\\\small
\begin{tabularx}{\textwidth}{|p{2.5cm}|p{2.5cm}|X|X|X|X|}
\hline
\multicolumn{6}{|X|}{\textbf{Nivel: 06 }}\\\hline\textbf{Materia} & \textbf{Docente} & \textbf{Estudiantes} & \textbf{Aprobados} & \textbf{Promedio} & \textbf{\%Supera el Promedio} \\ \hline
Control de Calidad I & AGREDA EGAS EYLEN AMANDA & 17 & 16 & 34.24 & 58.82 \%\\ \hline
Bacteriología Clínica II & CASTRO DEMERA DICKE ALEJANDRO & 22 & 19 & 31.41 & 54.55 \%\\ \hline
Bacteriología Clínica II & CASTRO DEMERA DICKE ALEJANDRO & 12 & 0 & 0 & 0.00 \%\\ \hline
Bacteriología Clínica II & CASTRO DEMERA DICKE ALEJANDRO & 10 & 0 & 0 & 0.00 \%\\ \hline
Bioestadística I & ACOSTA GANAN MICHAEL ANDRES & 26 & 20 & 30.65 & 57.69 \%\\ \hline
Genética y Biología Molecular & HIDALGO TAPIA COSME ENRIQUE & 29 & 29 & 35.17 & 44.83 \%\\ \hline
Diseño y Evaluación de Proyect & ESTUPIÑAN SANCHEZ ALFREDO SANDINO & 30 & 29 & 38.27 & 56.67 \%\\ \hline
Patología Clinica & CALDERON RUIZ PAOLA ALEXANDRA & 27 & 27 & 44.96 & 62.96 \%\\ \hline
Tutorías de Acompañamiento & ACOSTA GANAN MICHAEL ANDRES & 28 & 0 & 0 & 0.00 \%\\ \hline
\end{tabularx}

\vspace{1cm}
\section{Análisis de Rendimiento}
En el análisis del rendimiento académico del curso, se observa que la mayoría de las materias presentan una media del curso entre 30 y 40, lo que indica un buen nivel general de comprensión y dominio de los conceptos. Sin embargo, en algunas materias, se observe una disminución significativa en la nota media, como en caso de "Bacteriología Clínica II", donde se presentan tres secciones con resultados muy bajos, con nulas o apenas un aprobado.

En este sentido, es importante mencionar que la materia "Bacteriología Clínica II" presenta resultados muy por debajo del promedio general, lo que sugiere la necesidad de un seguimiento más estrecho y apoyo adicional para los estudiantes que cursan esta materia. Por otro lado, las materias como "Patología Clínica" y "Diseño y Evaluación de Proyectos" presentan resultados muy buenos, con un porcentaje de aprobación muy alto y una media del curso entre 40 y 45, lo que indica un buen nivel de comprensión y dominio de los conceptos.

En cuanto al promedio general del curso, se observa que el porcentaje de estudiantes que supera el promedio es alto, lo que indica una buena calidad general en el rendimiento académico. Sin embargo, también es importante destacar que algunos estudiantes no han aprobado algunas materias, por lo que es importante implementar estrategias de apoyo y seguimiento para asegurar que todos los estudiantes tengan las oportunidades necesarias para comprender y dominar los conceptos.

En conclusión, el curso presenta resultados generalmente buenos, pero también hay áreas que requieren un seguimiento más estrecho y apoyo adicional. Es importante implementar estrategias para apoyar a los estudiantes que presentan dificultades y para garantizar que todos los estudiantes tengan las oportunidades necesarias para comprender y dominar los conceptos.\\
\vspace{1cm}\\\small
\begin{tabularx}{\textwidth}{|p{2.5cm}|p{2.5cm}|X|X|X|X|}
\hline
\multicolumn{6}{|X|}{\textbf{Nivel: 07 }}\\\hline\textbf{Materia} & \textbf{Docente} & \textbf{Estudiantes} & \textbf{Aprobados} & \textbf{Promedio} & \textbf{\%Supera el Promedio} \\ \hline
Control de Calidad II & ZÚÑIGA SOSA EVELIN ALEXANDRA & 19 & 16 & 33.32 & 31.58 \%\\ \hline
Micología Médica & VIZCAINO ORDOÑEZ LAURA DEL ROCIO  & 22 & 20 & 33.73 & 45.45 \%\\ \hline
Micología Médica & VIZCAINO ORDOÑEZ LAURA DEL ROCIO  & 11 & 0 & 0 & 0.00 \%\\ \hline
Micología Médica & VIZCAINO ORDOÑEZ LAURA DEL ROCIO  & 11 & 0 & 0 & 0.00 \%\\ \hline
Bioestadística II & MERA CARRANZA JOSSELYN GENESIS & 16 & 15 & 40.69 & 56.25 \%\\ \hline
Prácticas Preprofesionales & ZÚÑIGA SOSA EVELIN ALEXANDRA & 22 & 0 & 0 & 0.00 \%\\ \hline
Prácticas de Servicio Comunita & ZÚÑIGA SOSA EVELIN ALEXANDRA & 12 & 12 & 45.58 & 58.33 \%\\ \hline
Enfermedades Tropicales & CASTRO DEMERA DICKE ALEJANDRO & 14 & 0 & 0 & 0.00 \%\\ \hline
Tutorías de Acompañamiento & RUEDA CASTILLO YAJAIRA MARILIN & 21 & 0 & 0 & 0.00 \%\\ \hline
Tutorías de Acompañamiento & RUEDA CASTILLO YAJAIRA MARILIN & 28 & 0 & 0 & 0.00 \%\\ \hline
\end{tabularx}

\vspace{1cm}
\section{Análisis de Rendimiento}
El análisis del rendimiento académico de los estudiantes de primer nivel de TICs revela resultados variados en relación con las materias estudiadas. En general, se puede observar que la mayoría de las materias presentan un número de aprobados inferior al total de estudiantes matriculados, lo que indica que el rendimiento académico no es homogéneo entre los estudiantes.

En el curso "Control de Calidad II" impartido por la docente Zúñiga Sosa Evelin Alexandra, se observations un pequeño número de estudiantes aprobados (16) en relación con el total de estudiantes matriculados (19), lo que sugiere que la dificultad de la materia fue considerable. De igual manera, en la materia "Prácticas Preprofesionales" también impartida por Zúñiga Sosa Evelin Alexandra, no hubo aprobados en el grupo de 22 estudiantes matriculados, lo que indica que la materia presentó déficits en el aprendizaje.

Por otro lado, la materia "Micrología Médica" impartida por Vizcaino Ordóñez Laura del Rocío presentó resultados más dispares entre los grupos, aunque en general el promedio del curso fue relativamente alto. En el grupo de 22 estudiantes, 20 aprobaron la materia, lo que significó un porcentaje de superación del promedio del 45.45\%. Sin embargo, en los grupos de 11 estudiantes, ninguno aprobó la materia.

En la materia "Bioestadística II" impartida por Mera Carranza Joselyn Genesis, se observó un porcentaje de aprobados relativamente alto, con 15 de 16 estudiantes aprobados, lo que significó un porcentaje de superación del promedio del 56.25\%.

En relación con la materia "Prácticas de Servicio Comunitario" impartida por Zúñiga Sosa Evelin Alexandra, se observó un resultado positivo, con 12 de 12 estudiantes aprobados, lo que significó un porcentaje de superación del promedio del 58.33\%. Sin embargo, en las materias "Enfermedades Tropicales" impartida por Castro Demera Dicke Alejandro y "Tutorías de Acompañamiento" impartida por Rueda Castillo Yajaira Marilin, no hubo aprobados en los grupos matriculados, lo que indica que estas materias presentaron dificultades en el aprendizaje.

En conclusión, el análisis del rendimiento académico de los estudiantes de primer nivel de TICs revela resultados variados en relación con las materias estudiadas. Mientras que algunas materias presentaron resultados positivos, otras presentaron dificultades en el aprendizaje. Estos resultados sugieren que es necesario implementar estrategias de apoyo y recursos adicionales para que los estudiantes puedan superar las dificultades y mejorar su rendimiento académico.\\
\vspace{1cm}\\\small
\begin{tabularx}{\textwidth}{|p{2.5cm}|p{2.5cm}|X|X|X|X|}
\hline
\multicolumn{6}{|X|}{\textbf{Nivel: 08 }}\\\hline\textbf{Materia} & \textbf{Docente} & \textbf{Estudiantes} & \textbf{Aprobados} & \textbf{Promedio} & \textbf{\%Supera el Promedio} \\ \hline
Integración Curricular & PEÑA ROSAS GLORIA DEL VALLE & 25 & 24 & 35.76 & 72.00 \%\\ \hline
Enfermedades Tropicales & CASTRO DEMERA DICKE ALEJANDRO & 26 & 26 & 34.23 & 34.62 \%\\ \hline
Enfermedades Tropicales & CASTRO DEMERA DICKE ALEJANDRO & 12 & 0 & 0 & 0.00 \%\\ \hline
Administración de Laboratorio & SALAZAR DONOSO ELIAS HUMBERTO & 15 & 15 & 44.87 & 66.67 \%\\ \hline
Toxicología & PEÑA ROSAS GLORIA DEL VALLE & 13 & 13 & 39.38 & 53.85 \%\\ \hline
Análisis y Validación de Datos & RUEDA CASTILLO YAJAIRA MARILIN & 14 & 12 & 32.64 & 57.14 \%\\ \hline
\end{tabularx}

\vspace{1cm}
\section{Análisis de Rendimiento}
El análisis del rendimiento académico de los estudiantes de primer nivel de TICs revela una situación general satisfactoria, con algunas excepciones. En primer lugar, la materia "Enfermedades Tropicales" enseñada por el docente Castro Demera Dickey Alejandro presenta un rendimiento bajo, con un número de aprobados muy pequeño (2 de 38 estudiantes) y un promedio bajo (34.62). Esta situación es especialmente preocupante en la materia que se enfoca en enfermedades tropicales, que requiere una comprensión profunda y práctica en los contenidos.

Por otro lado, las materias "Administración de Laboratorio" y "Toxicología" enseñadas, respectivamente, por los docentes Salazar Donoso Elias Humberto y Peña Rosas Gloria del Valle, presentan rendimientos más aceptables. En "Administración de Laboratorio", la mayoría de los estudiantes (15 de 15) aprobaron la asignatura y obtuvieron un promedio relativamente alto (44.87). En "Toxicología", 13 de 13 estudiantes aprobaron la asignatura y obtuvieron un promedio moderado (39.38).

La materia "Análisis y Validación de Datos" enseñada por el docente Rueda Castillo Yajaira Marilin presenta un rendimiento ligeramente decepcionante, con un número de aprobados relativamente pequeño (12 de 14) y un promedio bajo (32.64).

En general, el rendimiento académico de los estudiantes de primer nivel de TICs es decente, con algunas materias destacándose por sus buenos resultados y otras requiriendo un mayor esfuerzo para mejorar. Es importante que los docentes y los administradores de la institución brinden asistencia y orientación a los estudiantes que presentan dificultades en determinadas materias para asegurar que todos puedan alcanzar sus objetivos educativos.\\
\vspace{1cm}\\\begin{tabularx}{\textwidth}{|X|X|X|}
    \hline
    \textbf{ELABORADO POR:} & \textbf{REVISADO POR:} & \textbf{APROBADO POR:} \\ \hline
    Firma: & Firma: & Firma:\\
    &&\\
    &&\\
    &&\\ \hline
    \textbf{Nombre: Homero Velasteguí} & \textbf{Nombre: Manuel Nevarez} & \textbf{Nombre: Pablo Pico Valencia PhD.} \\ \hline
    \textbf{Cargo: Coordinador Carrera} & \textbf{Cargo: Consejo de Escuela} & \textbf{Cargo: Director Académico} \\ \hline
    \textbf{Fecha: 9/3/2024} & \textbf{Fecha: 9/3/2024} & \textbf{Fecha: 9/3/2024} \\ \hline
    \end{tabularx}
