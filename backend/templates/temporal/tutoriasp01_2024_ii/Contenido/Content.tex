\section{Sección 1}
Contenido de la sección 1
\subsection{Subsección 1.1}
Contenido de la subsección 1.1
\subsubsection{Subsubsección 1.1.1}
Contenido de la subsubsección 1.1.1
Este es un pálkjdjjhkajhdharrafo de texto que va después de la subsubsección.
\begin{table}[h]
\centering
\begin{tabularx}{\textwidth}{|X|X|X|}
\hline
\multicolumn{3}{|c|}{\textbf{Titulo largo para escribir}}\\\hline
\textbf{Header 1} & \textbf{Header 2} & \textbf{Header 3} \\ \hline
A1 & B1 & C1 \\ \hline
A2 & B2 & C2 \\ \hline
\end{tabularx}
\caption{Descripción de la tabla}
\end{table}
\begin{figure}[h!]
\centering
\includegraphics[width=0.8\textwidth]{path/to/image.jpg}
\caption{Figura 1. Ejemplo de imagen}
\end{figure}
\section{Sección 2}
Contenido de la sección 2
\section{Sección 2}
Contenido de la sección 2
\section{Sección 2}
Contenido de la sección 2

\begin{table}[h]
        \centering
        \begin{tabularx}{\textwidth}{|X|X|X|X|p{5cm}|}
        \hline
        \multicolumn{5}{|c|}{\textbf{Reporte de Calificaciones}}\\\hline
        \textbf{Materia} & \textbf{Docente} & \textbf{Nota 1} & \textbf{Nota 2} & \textbf{Resultado} \\ \hline
        Álgebra Lineal & José Luis Carvajal & 14 & 23 & 37 \\ \hline
        Matemática Básica & Ángel Anchundia & 26 & 1 & 27 \\ \hline
        Algoritmos y Pseudocódigo & Adrián Vargas & 28 & 21 & 49 \\ \hline
        Comunicación Oral & Jairon Caballero & 0 & 0 & 0 \\ \hline
        Tecnologías de la Información & Manuel Nevárez & 10 & 0 & 10 \\ \hline
        \end{tabularx}
        \caption{Tabla que muestra las calificaciones de los estudiantes en diversas materias}
        \end{table}