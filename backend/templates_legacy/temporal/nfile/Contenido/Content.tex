\small
\begin{tabularx}{\textwidth}{|p{2.5cm}|p{2.5cm}|X|X|X|X|}
\hline
\multicolumn{6}{|X|}{\textbf{Nivel: 01 }}\\\hline\textbf{Materia} & \textbf{Docente} & \textbf{Estudiantes} & \textbf{Aprobados} & \textbf{Promedio} & \textbf{\%Supera el Promedio} \\ \hline
Matemática Básica & BUSTOS VERA RHAY PABLO & 24 & 22 & 37.38 & 50.00 \%\\ \hline
Física y Laboratorio & NEVAREZ TOLEDO MANUEL ROGELIO & 24 & 24 & 37.5 & 54.17 \%\\ \hline
Algoritmos y Pseudocódigo & POSLIGUA FLORES KLEBER ROLANDO & 24 & 23 & 35.33 & 66.67 \%\\ \hline
Algebra Lineal & CARVAJAL CARVAJAL JOSE LUIS & 24 & 21 & 32.92 & 54.17 \%\\ \hline
Tecnologías de la Información  & RIVERA BONE CARLOS ALEJANDRO & 24 & 22 & 43.67 & 75.00 \%\\ \hline
Comunicación Oral y Escrita & CABALLERO MOREIRA JAIRON ADRIAN & 24 & 23 & 39.54 & 50.00 \%\\ \hline
\end{tabularx}

\vspace{1cm}
\section{Análisis de Rendimiento}
En el curso de primer nivel de TICs, se observó un desempeño académico variable en las diferentes materias. En Matemática Básica, se registra un total de 24 estudiantes, de los cuales 22 aprobaron el curso, lo que representa un porcentaje de aprobados del 91.67\%. El promedio del curso fue de 37.38, lo que sugiere que los estudiantes tienen habilidades numéricas y lógicas sólidas.

En Física y Laboratorio, el desempeño fue similar, con 24 estudiantes y 24 aprobados, lo que coincide con el porcentaje de aprobados del 100\%. El promedio del curso fue de 37.5, lo que indica que los estudiantes tienen una buena comprensión de los conceptos físicos y pueden aplicarlos efectivamente en ejercicios prácticos.

En Algoritmos y Pseudocódigo, se observó un ligero descenso en el desempeño, con 23 aprobados de 24 estudiantes, lo que representa un porcentaje de aprobados del 95.83\%. El promedio del curso fue de 35.33, lo que sugiere que los estudiantes necesitan una mayor práctica y familiarización con los conceptos de programación.

En Algebra Lineal, el desempeño fue más bajo, con 21 aprobados de 24 estudiantes, lo que representa un porcentaje de aprobados del 87.5\%. El promedio del curso fue de 32.92, lo que indica que los estudiantes enfrentan dificultades para comprender los conceptos de algebra lineal y necesitan una mayor explicación y retroalimentación.

En Tecnologías de la Información, el desempeño fue nuevamente bueno, con 22 aprobados de 24 estudiantes, lo que representa un porcentaje de aprobados del 91.67\%. El promedio del curso fue de 43.67, lo que sugiere que los estudiantes tienen habilidades desarrolladas en el uso de tecnologías de la información.

Finalmente, en Comunicación Oral y Escrita, se observó un desempeño relativamente constante, con 23 aprobados de 24 estudiantes, lo que representa un porcentaje de aprobados del 95.83\%. El promedio del curso fue de 39.54, lo que indica que los estudiantes tienen habilidades comunicativas desarrolladas y pueden expresarse de manera efectiva.

En general, se puede concluir que el curso de primer nivel de TICs es un desafío para los estudiantes, pero también presenta oportunidades para el crecimiento y el desarrollo académico. Es importante que los docentes y estudiantes trabajen juntos para superar los desafíos y alcanzar los objetivos del curso.\\
\vspace{1cm}\\\small
\begin{tabularx}{\textwidth}{|p{2.5cm}|p{2.5cm}|X|X|X|X|}
\hline
\multicolumn{6}{|X|}{\textbf{Nivel: 03 }}\\\hline\textbf{Materia} & \textbf{Docente} & \textbf{Estudiantes} & \textbf{Aprobados} & \textbf{Promedio} & \textbf{\%Supera el Promedio} \\ \hline
Redes I & VELASTEGUI IZURIETA HOMERO JAVIER & 16 & 16 & 35.81 & 62.50 \%\\ \hline
Estadística y Probabilidades & BUSTOS VERA RHAY PABLO & 16 & 16 & 41.19 & 50.00 \%\\ \hline
Estructura de Datos & SAYAGO HEREDIA JAIME PAUL & 15 & 10 & 31.33 & 46.67 \%\\ \hline
Lectura y Escritura Académica & RAMIREZ LOZADA HAYDEE  & 16 & 16 & 45.81 & 68.75 \%\\ \hline
Arquitectura de Computadores & CARVAJAL CARVAJAL JOSE LUIS & 16 & 15 & 37.06 & 56.25 \%\\ \hline
Jesucristo y la Persona de Hoy & QUINTERO ROSALES FRANCISCO JHONNY & 16 & 14 & 33.88 & 56.25 \%\\ \hline
\end{tabularx}

\vspace{1cm}
\section{Análisis de Rendimiento}
El análisis del rendimiento académico de los estudiantes de primer nivel de TICs muestra un panorama heterogéneo en cuanto a la calidad de las calificaciones obtenidas en las diferentes materias. En general, se puede destacar que la mayoría de los estudiantes han obtenido calificaciones altas en las materias de Redes I (35.81) y Lectura y Escritura Académica (45.81), lo que sugiere que han desarrollado habilidades significantes en estas áreas.

Por otro lado, se observa una tendencia a las calificaciones medias en materias como Estadística y Probabilidades (41.19) y Arquitectura de Computadores (37.06). Esta variabilidad en la calidad de las calificaciones puede ser explicada por la diversidad de habilidades y conocimientos que los estudiantes han adquirido en sus respectivas áreas de estudio.

Sin embargo, es importante destacar que las materias de Estructura de Datos (31.33) y Jesucristo y la Persona de Hoy (33.88) han obtenido calificaciones significativamente más bajas. Esto puede indicar que los estudiantes han encontrado desafíos en el aprendizaje de estos contenidos, lo que sugiere la necesidad de una revisión y ajuste en la estrategia de enseñanza y aprendizaje para mejorar la comprensión y retención de la información.

En términos de aprobación, se observa que la mayoría de los estudiantes han superado las materias, con un porcentaje de aprobación acumulado del 62.50. Sin embargo, es notorio que las materias de Estadística y Probabilidades y Jesucristo y la Persona de Hoy han obtenido tasas de aprobación más bajas, lo que sugiere la necesidad de apoyo adicional para los estudiantes que luchan con estos contenidos.

En general, el promedio del curso es de 38.55, lo que indica que los estudiantes han desarrollado habilidades significantes en un amplio rango de materias. Sin embargo, es importante destacar que la brecha entre las mejores y peores calificaciones es notable, lo que sugiere la necesidad de una revisión sistemática de la estrategia de enseñanza y aprendizaje para garantizar que todos los estudiantes tengan oportunidades iguales y equidad de acceso a la información y los recursos necesarios para alcanzar exitosamente.\\
\vspace{1cm}\\\small
\begin{tabularx}{\textwidth}{|p{2.5cm}|p{2.5cm}|X|X|X|X|}
\hline
\multicolumn{6}{|X|}{\textbf{Nivel: 05 }}\\\hline\textbf{Materia} & \textbf{Docente} & \textbf{Estudiantes} & \textbf{Aprobados} & \textbf{Promedio} & \textbf{\%Supera el Promedio} \\ \hline
Redes Inalambricas & VELASTEGUI IZURIETA HOMERO JAVIER & 17 & 17 & 41.65 & 52.94 \%\\ \hline
Desarrollo Basado En Plataform & SAYAGO HEREDIA JAIME PAUL & 18 & 16 & 35.5 & 61.11 \%\\ \hline
Base de Datos 2 & CARVAJAL CARVAJAL JOSE LUIS & 17 & 17 & 41.82 & 58.82 \%\\ \hline
Administracion de Sistemas Ope & PLATA CABRERA CARLOS SIMON & 17 & 17 & 48.35 & 58.82 \%\\ \hline
Arquitectura y Plataforma de S & SAYAGO HEREDIA JAIME PAUL & 17 & 17 & 39.59 & 58.82 \%\\ \hline
Tutorías de Acompañamiento & CHILA GARCIA KAREN CAROLINA & 1 & 0 & 0 & 0.00 \%\\ \hline
Tutorías de Acompañamiento & VELASTEGUI IZURIETA HOMERO JAVIER & 71 & 0 & 0 & 0.00 \%\\ \hline
Etica Personal y Socioambienta & BAUTISTA COTERA JAVIER GEOVANNY & 17 & 17 & 45.71 & 70.59 \%\\ \hline
\end{tabularx}

\vspace{1cm}
\section{Análisis de Rendimiento}
En el presente análisis de rendimiento académico de los estudiantes de primer nivel de TICs, se puede observar que los resultados varían según la materia. En general, se puede decir que la mayoría de los estudiantes han demostrado un buen desempeño académico, aunque hay algunas materias en las que se han observado resultados más heterogéneos.

La materia de "Redes Inalambricas" ha sido la que menos ha presentado dificultades, con un promedio del curso del 41.65 y un porcentaje de aprobados del 52.94\%. En contraste, la materia de "Tutorías de Acompañamiento" ha sido la que menos ha demostrado resultados positivos, con un promedio del curso de 0 y un porcentaje de aprobados del 0.00\%.

La materia de "Base de Datos 2" y la de "Administracion de Sistemas Operativos" han presentado resultados similares, con promedios del curso del 41.82 y 48.35 respectivamente, y porcentajes de aprobados del 58.82 y 58.82\% respectivamente.

Las materias de "Desarrollo Basado En Plataforma" y "Arquitectura y Plataforma de Servicios" han presentado resultados más moderados, con promedios del curso del 35.5 y 39.59 respectivamente, y porcentajes de aprobados del 61.11 y 58.82\% respectivamente.

Por último, la materia de "Etica Personal y Socioambiental" ha sido la que más ha logrado satisfacer a los estudiantes, con un promedio del curso del 45.71 y un porcentaje de aprobados del 70.59\%.

En general, se puede concluir que los estudiantes de primer nivel de TICs han demostrado un buen desempeño académico en algunas materias, aunque es importante destacar que hay algunas materias en las que se necesitan brindar más apoyo y estructura para lograr mejores resultados.\\
\vspace{1cm}\\\small
\begin{tabularx}{\textwidth}{|p{2.5cm}|p{2.5cm}|X|X|X|X|}
\hline
\multicolumn{6}{|X|}{\textbf{Nivel: 07 }}\\\hline\textbf{Materia} & \textbf{Docente} & \textbf{Estudiantes} & \textbf{Aprobados} & \textbf{Promedio} & \textbf{\%Supera el Promedio} \\ \hline
Gestion y Seguridad de Redes & VELASTEGUI IZURIETA HOMERO JAVIER & 13 & 13 & 42.54 & 46.15 \%\\ \hline
Practicas Pre Profesionales & CARVAJAL CARVAJAL JOSE LUIS & 9 & 9 & 48.56 & 88.89 \%\\ \hline
Prácticas de Servicio a la Com & SAYAGO HEREDIA JAIME PAUL & 14 & 14 & 44.57 & 64.29 \%\\ \hline
Interacción Humano Computador & PICO VALENCIA PABLO ANTONIO & 13 & 12 & 36.85 & 61.54 \%\\ \hline
Herramientas y Técnicas de Cib & VELASTEGUI IZURIETA HOMERO JAVIER & 13 & 13 & 37 & 30.77 \%\\ \hline
Diseño y Evaluación de Proyect & QUIÑONEZ KU VICTOR XAVIER & 13 & 13 & 37.46 & 46.15 \%\\ \hline
Integración Curricular & SINCHI SINCHI HUGO FERNANDO & 2 & 1 & 22 & 50.00 \%\\ \hline
\end{tabularx}

\vspace{1cm}
\section{Análisis de Rendimiento}
En el curso de primer nivel de TICs, se han presentado variados resultados en cuanto al rendimiento académico de los estudiantes. De acuerdo a los datos proporcionados, es posible observar que la mayor cantidad de estudiantes se han matriculado en la materia "Gestión y Seguridad de Redes" con 13 estudiantes, seguida por "Práctica Pre Profesionales" y "Diseño y Evaluación de Proyectos" con 14 y 13 estudiantes respectivamente. 

En cuanto al rendimiento, se puede ver que la materia "Prácticas de Servicio a la Comunidad" ha logrado el mayor porcentaje de aprobados con un 88.89\%, seguida de "Práctica Pre Profesionales" con un 76.92\%. En cuanto al promedio del curso, se puede observar que no hay una materia con un promedio superior al resto, lo que sugiere que los estudiantes han tenido una desempeñado de manera constante en todas las materias.

Sin embargo, se pueden observar algunas materias que han tenido resultados más bajos en cuanto al rendimiento. La materia "Herramientas y Técnicas de Programación" ha obtenido el menor promedio con un 30.77\%, lo que sugiere que los estudiantes tienen necesidad de recibir mayor apoyo y capacitación en esta área. Por otro lado, la materia "Interacción Humano Computador" ha obtenido un promedio de 36.85, lo que también sugiere que necesita mejorar.

En cuanto al porcentaje de estudiantes que superan el promedio, se puede ver que solo dos materias han logrado este objetivo: "Practicas de Servicio a la Comunidad" con un 64.29\% y "Integración Curricular" con un 50\%. Esto sugiere que los estudiantes necesitan trabajar más para lograr una mayor comprensión de los conceptos y habilidades relacionadas con las materias.

En resumen, el análisis del rendimiento académico de los estudiantes de primer nivel de TICs revela que aunque hay algunas materias que han logrado buenos resultados, también hay algunas area que necesitan mejorar. Es importante que los docentes y estudiantes trabajen juntos para identificar áreas de oportunidad y desarrollar estrategias para mejorar la comprensión y los resultados en cada materia.\\
\vspace{1cm}\\\begin{tabularx}{\textwidth}{|X|X|X|}
    \hline
    \textbf{ELABORADO POR:} & \textbf{REVISADO POR:} & \textbf{APROBADO POR:} \\ \hline
    Firma: & Firma: & Firma:\\
    &&\\
    &&\\
    &&\\ \hline
    \textbf{Nombre: Homero Velasteguí} & \textbf{Nombre: Manuel Nevarez} & \textbf{Nombre: Pablo Pico Valencia PhD.} \\ \hline
    \textbf{Cargo: Coordinador Carrera} & \textbf{Cargo: Consejo de Escuela} & \textbf{Cargo: Director Académico} \\ \hline
    \textbf{Fecha: 9/3/2024} & \textbf{Fecha: 9/3/2024} & \textbf{Fecha: 9/3/2024} \\ \hline
    \end{tabularx}
