% Plantilla Jinja2: completar variables y renderizar a Preambulo/datos.tex

\newif\ifshowfirmas
\showfirmastrue

\newif\ifshowbibliografia
\showbibliografiafalse
% Si make.sh genera Preambulo/bibflag.tex, se usa como override automático
\IfFileExists{Preambulo/bibflag.tex}{\input{Preambulo/bibflag}}{}

% Parámetros de firma (definidos aquí; sobrescribir con \def o \renewcommand si es necesario)
\def\firmaElaboradoToken{!-1804326534-!}
\def\firmaRevisadoToken{!-0000000000-!}
\def\firmaAprobadoToken{!-0000000000-!}
\def\nombreElaborado{Mgt. Velasteguí Izurieta Homero Javier}
\def\nombreRevisado{Mgt. Carvajal Carvajal José Luis}
\def\nombreAprobado{Mgt. Puente Holguín Washington David}
\def\cargoElaborado{Coordinador de Carrera}
\def\cargoRevisado{Responsable de Aseguramiento de la Calidad}
\def\cargoAprobado{Director de Escuela}
\def\fechaElaborado{\mydate\today}
\def\fechaRevisado{\mydate\AdvanceDate[2]\today}
\def\fechaAprobado{\mydate\AdvanceDate[4]\today}

\def\semestre{2025-I}
\def\titulo{\textbf{INFORME ESTUDIANTIL}}
\def\carrera{Carrera de Ingeniería en Tecnologías de la Información}
\def\coordinador{Velasteguí Izurieta Homero Javier}

% Layouts
\def\layoutDefaultName{portrait}


\def\layoutOptionsportrait{a4paper,portrait, total={160mm,247mm}, left=25mm, top=0mm}



\def\layoutOptionslandscape{a4paper,landscape, total={289mm,170mm}, left=4mm, top=0mm}

