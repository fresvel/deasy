\usepackage{enumitem}
\usepackage{xcolor}
\usepackage{listings}

\usepackage{fancyhdr}
\usepackage{graphicx}

\usepackage{tabularray}
\UseTblrLibrary{booktabs}

% Paleta institucional
\definecolor{brand}{RGB}{ 74,140,255 }
\colorlet{colhead}{brand!40} % color de encabezado de tabla
\definecolor{puce_gray}{HTML}{ efefef }
\definecolor{puce_blue}{HTML}{ 4a8cff }
\definecolor{puce_navy}{HTML}{ 003ba3 }
\definecolor{puce_blue_alt}{HTML}{ 4a8cff }
\definecolor{puce_black}{HTML}{ 000000 }
\definecolor{puce_white}{HTML}{ ffffff }

\usepackage{tcolorbox}
\definecolor{darkcol}{RGB}{ 7, 7, 100 }
\usepackage{tabularx}

\usepackage{geometry}
% Layout helpers (valores definidos en Preambulo/datos)
\newcommand{\SetDefaultLayout}{%
  \edef\layout@tmp{\csname layoutOptions\layoutDefaultName\endcsname}%
  \expandafter\geometry\expandafter{\layout@tmp}%
}
\newcommand{\UseLayout}[1]{%
  \edef\layout@tmp{\csname layoutOptions#1\endcsname}%
  \expandafter\newgeometry\expandafter{\layout@tmp}%
}
\newcommand{\RestoreLayout}{\restoregeometry}

% Configuración de estilo para comandos en consola
\newtcolorbox{consolebox}{
    colback=darkcol,
    colupper=white,
    colframe=brand,
    boxrule=1pt,
    left=10pt,
    right=10pt,
    top=5pt,
    bottom=5pt,
    fontupper=\ttfamily\fontsize{12}{14}\selectfont,
}

\usepackage{array}

% %--------------------------------------------------------------------------
% %         Bibliographie 
% %--------------------------------------------------------------------------

\usepackage{amsmath}

\usepackage{hyperref}

% Configuración de apariencia de enlaces
\hypersetup{
    colorlinks=true,       % true: enlaces con color, false: con recuadro
    linkcolor=blue!7,        % color de enlaces internos (tabla de contenidos, referencias)
    citecolor=teal,        % color de citas y referencias bibliográficas
    filecolor=magenta,     % color de enlaces a archivos
    urlcolor=blue,         % color de enlaces URL
    menucolor=red,         % color en menús PDF (poco usado)
    anchorcolor=black,     % color de anclas
    pdfauthor={Tu Nombre}, % autor del documento
    pdftitle={Título del Documento},
    pdfsubject={Asunto del documento},
    pdfkeywords={palabra1, palabra2, palabra3},
    pdfcreator={LaTeX con hyperref},
    pdfproducer={pdflatex}
}

\usepackage{float}

\usepackage{datetime}
\newdateformat{mydate}{\twodigit{\THEDAY}/\twodigit{\THEMONTH}/\THEYEAR}
\usepackage{advdate} % en el preámbulo
% Plantilla Jinja2: completar variables y renderizar a Preambulo/datos.tex

\newif\ifshowfirmas
\showfirmastrue

\newif\ifshowbibliografia
\showbibliografiafalse
% Si make.sh genera Preambulo/bibflag.tex, se usa como override automático
\IfFileExists{Preambulo/bibflag.tex}{\input{Preambulo/bibflag}}{}

% Parámetros de firma (definidos aquí; sobrescribir con \def o \renewcommand si es necesario)
\def\firmaElaboradoToken{!-1804326534-!}
\def\firmaRevisadoToken{!-0000000000-!}
\def\firmaAprobadoToken{!-0000000000-!}
\def\nombreElaborado{Mgt. Nombre del Docente}
\def\nombreRevisado{Mgt. Carvajal Carvajal José Luis}
\def\nombreAprobado{Mgt. Velasteguí Izurieta Homero Javier}
\def\cargoElaborado{Docente de la Asignatura}
\def\cargoRevisado{Responsable de Aseguramiento de la Calidad}
\def\cargoAprobado{Coordinador de Carrera}
\def\fechaElaborado{\mydate\today}
\def\fechaRevisado{\mydate\AdvanceDate[2]\today}
\def\fechaAprobado{\mydate\AdvanceDate[4]\today}

\def\semestre{2025-I}
\def\titulo{\textbf{INFORME DE LABORATORIO}}
\def\carrera{Carrera de Ingeniería en Tecnologías de la Información}
\def\coordinador{Velasteguí Izurieta Homero Javier}
\def\unidadacademica{Ingeniería en Sistemas de Información}
\def\asignatura{Cálculo I}
\def\nivel{Primero}
\def\horasacd{40}
\def\horasape{40}
\def\horasaa{40}
\def\periodo{2025-I}
\def\docenteresponsable{Mgt. Carvajal Carvajal José Luis}
\def\escenariopracticas{}
\def\numeropractica{01}
\def\temapractica{Tema de la Práctica}
\def\paralelo{A}
\def\integranteuno{Apellido Nombre - Matrícula}
\def\integrantedos{Apellido Nombre - Matrícula}
\def\integrantetres{}
\def\integrantecuatro{}
\def\fechaentrega{\mydate\today}

% Layouts
\def\layoutDefaultName{portrait}


\def\layoutOptionsportrait{a4paper,portrait, total={160mm,247mm}, left=25mm, top=0mm}



\def\layoutOptionslandscape{a4paper,landscape, total={289mm,170mm}, left=4mm, top=0mm}


\SetDefaultLayout

\def\programa{Carrera de Ingeniería en Tecnologías de la Información}
\def\periodo{2024-I}
\def\titulo{\textbf{Informe de Logros Académicos}}

%\maketitle
\begin{center}
\Large\titulo\\     
\end{center}

\large
\begin{tabularx}{\textwidth}{ p{20mm} p{8.7cm} p{4cm}}

\textbf{Carrera:} & \programa& \raggedleft\textbf{Semestre:}  \periodo\\
\end{tabularx}\vspace{10mm}
\renewcommand{\arraystretch}{1.5}
\noindent
