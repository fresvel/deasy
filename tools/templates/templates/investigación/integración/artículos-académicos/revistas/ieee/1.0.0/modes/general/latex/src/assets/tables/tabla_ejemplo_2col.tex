\begin{table*}[!t]
\centering
\begin{longtblr}[
  caption = {Antecedentes de Investigación sobre Tecnologías de Alerta para Personas Sordas},
  label = {tab:ejemplo_dos_columna},
]{
  colspec = {
    X[0.10,l]
    X[0.30,l]
    X[0.30,l]
  },
  width = \textwidth,
  rowhead = 1,
  hlines,
  vlines,
  row{1-Z}={font=\footnotesize}
}
\textbf{Autor/es y Año} & \textbf{Título del Artículo} & \textbf{Resultados Relevantes para esta Investigación} \\

Fletcher, M. D. (2021) &
Using haptic stimulation to enhance auditory perception in hearing-impaired listeners & Menciona que la estimulación háptica mejora la localización de amenazas. \\

Dim, C. A. \textit{et al.} (2022) &
Alert systems to hearing-impaired people: a systematic review &
Revela brecha investigativa (<10\% enfocado en sordos vs. visual). Critica la falta de validación en entornos reales. \\

Rashid, N. A. A. \textit{et al.} (2024) &
Assistive Technology for The Deaf: A Literature Review &
Destaca el alto costo de los wearables. \\

Goodman, S. \textit{et al.} (2020) &
Evaluating smartwatch-based sound feedback for deaf and hard-of-hearing users across contexts & La vibración constante resulta molesta socialmente. Propone vibración solo para captar atención y pantalla para detalles. \\

Jain, D. \textit{et al.} (2020) &
HomeSound: An Iterative Field Deployment of an In-Home Sound Awareness System... & Confirma visualización de sonidos mejora la conciencia del entorno, pero errores y alertas excesivas generan desconfianza. \\

Kudrinko, K. \textit{et al.} (2020) &
Wearable Sensor-Based Sign Language Recognition: A Comprehensive Review & Concluye que los wearables ofrecen una portabilidad real. Diseño de guantes actuales son costosos, voluminosos e incómodos. \\

Qahtan, S. \textit{et al.} (2023) &
A comparative study of evaluating and benchmarking sign language recognition system... & Concluye que aún no existe un sistema que cubra todas las características deseables. \\

\end{longtblr}
\end{table*}


\begin{table*}[!t]
\centering
\begin{longtblr}[
  caption = {Antecedentes de Investigación sobre Tecnologías de Alerta para Personas Sordas},
  label = {tab:ejemplo_dos_columna2},
]{
  colspec = {
    X[0.10,l]
    X[0.30,l]
    X[0.30,l]
  },
  width = \textwidth,
  rowhead = 1,
  hlines,
  vlines,
  row{1-Z}={font=\footnotesize}
}
\textbf{Autor/es y Año} & \textbf{Título del Artículo} & \textbf{Resultados Relevantes para esta Investigación} \\

Fletcher, M. D. (2021) &
Using haptic stimulation to enhance auditory perception in hearing-impaired listeners & Menciona que la estimulación háptica mejora la localización de amenazas. \\

Dim, C. A. \textit{et al.} (2022) &
Alert systems to hearing-impaired people: a systematic review &
Revela brecha investigativa (<10\% enfocado en sordos vs. visual). Critica la falta de validación en entornos reales. \\

Rashid, N. A. A. \textit{et al.} (2024) &
Assistive Technology for The Deaf: A Literature Review &
Destaca el alto costo de los wearables. \\

Goodman, S. \textit{et al.} (2020) &
Evaluating smartwatch-based sound feedback for deaf and hard-of-hearing users across contexts & La vibración constante resulta molesta socialmente. Propone vibración solo para captar atención y pantalla para detalles. \\

Jain, D. \textit{et al.} (2020) &
HomeSound: An Iterative Field Deployment of an In-Home Sound Awareness System... & Confirma visualización de sonidos mejora la conciencia del entorno, pero errores y alertas excesivas generan desconfianza. \\

Kudrinko, K. \textit{et al.} (2020) &
Wearable Sensor-Based Sign Language Recognition: A Comprehensive Review & Concluye que los wearables ofrecen una portabilidad real. Diseño de guantes actuales son costosos, voluminosos e incómodos. \\

Qahtan, S. \textit{et al.} (2023) &
A comparative study of evaluating and benchmarking sign language recognition system... & Concluye que aún no existe un sistema que cubra todas las características deseables. \\

\end{longtblr}
\end{table*}


