\section{Results}

% ---------------------------------------------------------
\subsection{Study Selection}

La búsqueda inicial identificó un total de XXX registros. Tras la eliminación de duplicados y la aplicación de los criterios de elegibilidad en las fases de cribado y revisión a texto completo, finalmente se incluyeron XX estudios en la presente revisión sistemática.

% --- FIGURA: DIAGRAMA PRISMA -----------------------------
\begin{figure}[ht]
\centering
% \includegraphics[width=\textwidth]{prisma_diagram.pdf}
\caption{Diagrama PRISMA 2020 del flujo de selección de estudios.}
\label{fig:prisma}
\end{figure}

% --- TABLA: RAZONES DE EXCLUSIÓN -------------------------
\begin{table}[ht]
\centering
\caption{Razones de exclusión de los estudios tras la revisión a texto completo.}
\label{tab:razones_exclusion}
\begin{tabular}{lll}
\toprule
Categoría & Referencias Excluidas & Número de Estudios \\
\midrule
% Ejemplo:
% No aborda IA & \cite{ref1, ref2, ref3} & 3 \\
% Dataset insuficiente & \cite{ref4, ref5} & 2 \\
\bottomrule
\end{tabular}
\end{table}

% ---------------------------------------------------------
\subsection{Characteristics of Included Studies}

Aquí se describe la evidencia general de los estudios incluidos:(aquí va la estadística descriptiva)
- año y origen de publicación,
- tipos de modelos de IA empleados,
- modalidades de imagen analizadas,
- tamaños de dataset,
- diseños experimentales,
- métricas principales.

% --- TABLA: CARACTERÍSTICAS GENERALES --------------------
\begin{table}[ht]
\centering
\caption{Características principales de los estudios incluidos.}
\label{tab:characteristics}
% Puedes usar longtblr aquí
\end{table}

% ---------------------------------------------------------
\subsection{Synthesis of Results According to Research Questions}

A continuación se presentan los hallazgos organizados en función de las RQs establecidas.

% ----------------- RQ1 -----------------------------------
\subsubsection*{RQ1. ¿Qué modelos de Inteligencia Artificial se utilizan para el diagnóstico de cáncer de mama basado en imágenes?}

(Describe tendencias, arquitecturas comunes, evolución temporal, casos destacados.)

% ----------------- RQ2 -----------------------------------
\subsubsection*{RQ2. ¿Qué modalidades de imagen son empleadas con mayor frecuencia?}

(Comparación: mamografía, RM, ecografía, tomosíntesis, histología digital, etc.)

% ----------------- RQ3 -----------------------------------
\subsubsection*{RQ3. ¿Qué métricas de desempeño se reportan en los estudios?}

(AUC, exactitud, sensibilidad, F1-score, IoU, Dice…  
Puedes incluir una tabla o gráfica descriptiva.)

% ----------------- RQ4 -----------------------------------
\subsubsection*{RQ4. ¿Qué bases de datos o repositorios se utilizan con mayor frecuencia?}

(Describe INbreast, DDSM, MIAS, CBIS-DDSM, BreakHis, datasets privados.)

% ----------------- RQ5 -----------------------------------
\subsubsection*{RQ5. ¿Qué limitaciones, desafíos o brechas identifican los estudios?}

(Problemas de generalización, falta de validación externa, sesgos, datasets pequeños.)

% ---------------------------------------------------------
\subsection{Additional Analyses (Optional)}

Aquí puedes incluir —si deseas— análisis adicionales:
- correlación calidad vs desempeño,
- análisis temporal (tendencias por año),
- mapas conceptuales,
- clasificación por familias de arquitecturas (CNN, ViT, híbridos),
- síntesis temática.

% ---------------------------------------------------------
\subsection{Summary of Findings}

Un resumen final que transforme la síntesis en afirmaciones concluyentes
sobre el estado del arte.

