\section{Introducción}

En esta sección, el docente debe contextualizar la práctica de laboratorio dentro de la asignatura y del plan de estudios, estableciendo su sentido formativo y su relación con los contenidos abordados en clase. La introducción no debe desarrollar teoría ni describir el procedimiento del laboratorio; su función es orientar académicamente la práctica.

\subsection*{Estructura obligatoria de la introducción}

La introducción debe organizarse, como mínimo, en los siguientes componentes, presentados de manera integrada y coherente.

\paragraph{1. Contexto de la práctica}
El docente debe indicar a qué asignatura pertenece la práctica, qué unidad o tema del curso refuerza y el tipo de laboratorio que se desarrollará (experimental, simulación, implementación, análisis u otro).  

Estructura sugerida:
\begin{quote}
Esta práctica de laboratorio se enmarca en la asignatura \textbf{[nombre]}, dentro de la unidad \textbf{[unidad o tema]}, y corresponde a una actividad de tipo \textbf{[tipo de laboratorio]}.
\end{quote}

\paragraph{2. Propósito formativo}
El docente debe describir el propósito general de la práctica, explicando para qué se realiza y qué aporta al proceso de aprendizaje del estudiante. Esta descripción debe centrarse en el logro formativo y no en las acciones operativas.

Estructura sugerida:
\begin{quote}
El propósito de esta práctica es \textbf{[finalidad formativa]}, permitiendo al estudiante \textbf{[logro esperado en términos de aplicación, análisis o verificación]}.
\end{quote}

\paragraph{3. Alineación con los resultados de aprendizaje}
El docente debe establecer explícitamente la relación entre la práctica y los resultados de aprendizaje de la asignatura. Estos resultados no deben reformularse, sino referenciarse o describirse de manera consistente con el syllabus.

Estructura sugerida:
\begin{quote}
La práctica contribuye al desarrollo de los siguientes resultados de aprendizaje de la asignatura: \textbf{[referencia o descripción breve de los resultados]}.
\end{quote}


\paragraph{4. Alcance y supuestos}
El docente debe delimitar el alcance de la práctica, indicando qué aspectos serán abordados y cuáles quedan fuera de su objetivo. Asimismo, se deben explicitar los conocimientos previos que se asume posee el estudiante.

Estructura sugerida:
\begin{quote}
Esta práctica aborda \textbf{[aspectos cubiertos]} y no contempla \textbf{[aspectos excluidos]}. Se asume que el estudiante cuenta con conocimientos previos en \textbf{[prerrequisitos]}.
\end{quote}

\paragraph{5. Fundamento teórico de soporte}
Dado que la guía no contempla una sección específica de teoría, el docente debe indicar en esta sección las principales fuentes teóricas que sustentan la práctica. Estas fuentes se presentan como respaldo conceptual.  Puede ser una sección o subsección no numerada (\\sección*{})

Estructura sugerida:
\begin{quote}
La práctica se sustenta en los fundamentos teóricos desarrollados en \textbf{[libros, artículos, normas o documentación técnica]}, los cuales constituyen la base conceptual para la ejecución del laboratorio.
\end{quote}

Se recomienda incluir entre tres y cinco referencias clave, evitando definiciones extensas o explicaciones detalladas.




