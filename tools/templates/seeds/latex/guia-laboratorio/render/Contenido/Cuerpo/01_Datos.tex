% Datos generales de la práctica
{
\centering
\SetTblrTemplate{caption}{empty}

\begin{tblr}{
  colspec={X[l,m,35] X[l,m,42] X[l,m,21]},
  hlines,vlines,
  % Small caps para toda la tabla
  row{1-Z} = {font=\small\scshape},
}
% NOTA: Evita \textbf{\textsc{...}} porque muchas fuentes no tienen bold small caps.
% Usa \bfseries dentro del contexto \scshape ya aplicado a la tabla.

{\bfseries UNIDAD Académica} & \SetCell[c=2]{l} \unidadacademica & \\
{\bfseries Asignatura}       & \asignatura                      & {\bfseries Nivel:} \nivel \\
{\bfseries Número de horas}  & \SetCell[c=2]{l}
  {\bfseries ACD:} \horasacd\quad
  {\bfseries APE:} \horasape\quad
  {\bfseries AA:}  \horasaa
  & {\bfseries Duración:} 4 \\
{\bfseries Periodo}          & \SetCell[c=2]{l} \periodo        & \\
{\bfseries Docente responsable} & \SetCell[c=2]{l} \docenteresponsable & \\
{\bfseries Escenario de prácticas} & \SetCell[c=2]{l} \escenariopracticas & \\

\end{tblr}
}


\begin{center}
  \textbf{Práctica N°: }\numeropractica \\[0.25cm]
  \textsc{\temapractica}

\end{center}

\noindent
\rule{\linewidth}{1pt}

