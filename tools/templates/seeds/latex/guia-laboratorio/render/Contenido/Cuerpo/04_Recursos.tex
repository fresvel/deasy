\section{Recursos}

En esta sección, el docente debe especificar de manera clara y organizada todos los recursos necesarios para la correcta ejecución de la práctica de laboratorio. La finalidad de esta sección es garantizar la replicabilidad de la práctica y facilitar la planificación logística, evitando ambigüedades sobre los insumos requeridos.

\subsection*{Clasificación de los recursos}

El docente debe organizar los recursos en categorías claramente diferenciadas, según la naturaleza de la práctica. No es obligatorio utilizar todas las categorías; se deben incluir únicamente aquellas que apliquen.

\paragraph{1. Equipamiento e infraestructura}
Se deben detallar los equipos, dispositivos, instrumentos o espacios físicos necesarios para el desarrollo del laboratorio. Cuando aplique, se debe indicar la cantidad mínima requerida y las condiciones de uso.

Estructura sugerida:
\begin{quote}
Equipo o infraestructura: \textbf{[nombre]} \\
Cantidad mínima: \textbf{[número o proporción]} \\
Condiciones de uso: \textbf{[restricciones, seguridad, disponibilidad]}
\end{quote}

\paragraph{2. Materiales e insumos}
Se deben listar los materiales, componentes o insumos consumibles necesarios para la práctica, indicando, cuando sea pertinente, características relevantes como tipo, especificación o presentación.

Estructura sugerida:
\begin{quote}
Material o insumo: \textbf{[nombre]} \\
Especificación relevante: \textbf{[tipo, modelo, pureza, versión, etc.]} \\
Observaciones: \textbf{[consideraciones especiales]}
\end{quote}

\paragraph{3. Software y herramientas digitales}
Se deben especificar las aplicaciones, lenguajes, plataformas, librerías o herramientas digitales requeridas para la práctica. Es recomendable indicar versiones mínimas o equivalencias aceptadas.

Estructura sugerida:
\begin{quote}
Herramienta o software: \textbf{[nombre]} \\
Versión mínima o recomendada: \textbf{[versión]} \\
Alternativas aceptadas: \textbf{[cuando aplique]}
\end{quote}

\paragraph{4. Recursos de información}
Se deben indicar los documentos, manuales, guías técnicas, conjuntos de datos o repositorios que el estudiante utilizará durante la práctica. Estos recursos deben estar alineados con las fuentes mencionadas en la introducción.

Estructura sugerida:
\begin{quote}
Recurso: \textbf{[nombre o referencia]} \\
Tipo: \textbf{[manual, artículo, dataset, repositorio, norma]} \\
Uso en la práctica: \textbf{[consulta, referencia, insumo de trabajo]}
\end{quote}

\subsection*{Recursos mínimos y equivalentes}

El docente debe distinguir claramente entre los recursos mínimos indispensables para ejecutar la práctica y aquellos recursos equivalentes que pueden emplearse como alternativas sin afectar los objetivos del laboratorio.

Estructura sugerida:
\begin{quote}
Recursos mínimos: \textbf{[lista de recursos indispensables]} \\
Recursos equivalentes o alternativos: \textbf{[opciones aceptadas]}
\end{quote}