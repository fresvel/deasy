\section{Rúbrica de evaluación}

En esta sección, el docente debe definir los criterios de evaluación de la práctica de laboratorio mediante una rúbrica claramente estructurada. La rúbrica debe establecer los criterios a evaluar, los niveles de desempeño y la escala de valoración asociada, permitiendo una evaluación objetiva, transparente y coherente con el enfoque formativo de la práctica.

Los criterios de evaluación deben estar directamente vinculados con los objetivos definidos para el laboratorio y con las evidencias generadas durante su desarrollo. Cada criterio debe evaluar un aspecto específico del desempeño del estudiante y no una combinación de varios elementos.

La rúbrica debe incluir descriptores claros para cada nivel de logro. Estos descriptores deben indicar de manera explícita qué se espera del estudiante en cada nivel, evitando formulaciones ambiguas o subjetivas.

De manera general, la rúbrica de evaluación debe seguir la estructura del siguiente ejemplo.\\

\noindent
\begin{bodytblr}{
  colspec = { X[l,m,2.0] Q[c,m,1.6cm] X[l,m,1.7] X[l,m,1.7] X[l,m,1.7] X[l,m,1.7]},
}
Criterio de evaluación &
{Peso \\ (pts.)} &
{Alcanzado \\ (90--100)\% } &
{Medianamente \\ (80--89)\% } &
{Parcialmente \\ (60--79)\% } &
{No alcanzado \\ (0--59)\% } \\

[RdA\_x] Criterio\_1 & x pts & Descriptor A & Descriptor B & Descriptor C & Descriptor D \\
[RdA\_x] Criterio\_2 & x pts & Descriptor A & Descriptor B & Descriptor C & Descriptor D \\
[RdA\_x] Criterio\_3 & x pts & Descriptor A & Descriptor B & Descriptor C & Descriptor D \\
\end{bodytblr}


\subsection*{Estructura de los criterios de evaluación}

Cada criterio de evaluación debe redactarse como un enunciado breve que describa un aspecto evaluable del desempeño del estudiante. El criterio debe responder a la pregunta: \emph{¿qué se evalúa?}

Los criterios deben:
\begin{itemize}
  \item Estar alineados con uno o más objetivos de la práctica.
  \item Evaluar un solo aspecto del desempeño.
  \item Ser observables a partir de evidencias concretas.
\end{itemize}

Estructura sugerida del criterio:
\begin{quote}
\textbf{Acción o desempeño evaluado} + \textbf{objeto} + \textbf{condición o contexto}.
\end{quote}

Ejemplos estructurales:
\begin{quote}
Configuración correcta de \textbf{[sistema/herramienta]} según \textbf{[parámetros definidos]}.\\
Análisis de \textbf{[resultados/datos]} considerando \textbf{[criterios establecidos]}.
\end{quote}



\subsection*{Estructura de los descriptores de logro}

Los descriptores describen el nivel de desempeño del estudiante para un criterio específico. Todos los descriptores deben partir de una misma base conceptual y diferenciarse por el nivel cognitivo alcanzado o por la complejidad del contexto en el que se evidencia el desempeño.

Para la redacción de los descriptores, el docente debe considerar los siguientes principios:

\begin{itemize}
  \item El nivel de aprobación constituye la base mínima aceptable del criterio.
  \item Los niveles superiores se construyen elevando el verbo según la taxonomía de Bloom (por ejemplo: aplicar $\rightarrow$ analizar $\rightarrow$ evaluar).
  \item Alternativamente, se puede mantener el mismo verbo e incrementar la complejidad del contexto, las restricciones o el grado de autonomía del estudiante.
\end{itemize}

Estructura sugerida del descriptor:
\begin{quote}
\textbf{Verbo observable} + \textbf{objeto evaluado} + \textbf{condición o nivel de complejidad}.
\end{quote}


\subsection*{Criterios para diferenciar los niveles de logro}

\paragraph{Nivel alcanzado (90--100)}
El descriptor debe reflejar un desempeño de nivel cognitivo alto (analizar, evaluar, validar, optimizar) o la correcta aplicación del criterio en un contexto complejo, con autonomía y sin errores relevantes.

\paragraph{Nivel medianamente alcanzado (80--89)}
El descriptor debe reflejar un desempeño adecuado, normalmente asociado a verbos como aplicar o implementar, con resultados correctos pero con limitaciones menores o menor profundidad de análisis.

\paragraph{Nivel parcialmente alcanzado (60--79)}
El descriptor debe reflejar un desempeño básico, centrado en la ejecución mínima del criterio, con errores conceptuales o procedimentales que no impiden completamente el logro, pero sí limitan su calidad.

\paragraph{Nivel no alcanzado (0--59)}
El descriptor debe indicar la ausencia del logro esperado, errores críticos, incompletitud de la evidencia o la imposibilidad de demostrar el criterio evaluado.