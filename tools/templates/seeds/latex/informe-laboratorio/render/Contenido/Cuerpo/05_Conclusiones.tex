\section{Conclusiones}

En esta sección, el estudiante debe sintetizar los hallazgos principales de la práctica, vinculándolos directamente con los objetivos planteados. Las conclusiones no deben incluir información nueva ni repetir textualmente los resultados, sino interpretar su significado académico.

\subsection*{Recomendaciones para redactar conclusiones}

\begin{itemize}
  \item Presente entre 2 y 4 conclusiones claras y numeradas o separadas en párrafos.
  \item Relacione cada conclusión con un objetivo específico o con un resultado clave.
  \item Evite conclusiones genéricas; use evidencia de los resultados.
\end{itemize}

Estructura sugerida:
\begin{quote}
A partir de los resultados, se concluye que \textbf{[hallazgo principal]} porque \textbf{[evidencia]}.
\end{quote}