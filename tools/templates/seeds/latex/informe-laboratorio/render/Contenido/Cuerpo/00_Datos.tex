% Datos generales del informe de laboratorio (estudiantes)
{
\centering
\SetTblrTemplate{caption}{empty}

\begin{tblr}{
  colspec={X[l,m,0.35] X[l,m,0.325] X[l,m,0.325]},
  hlines,vlines,
  row{1-Z} = {font=\small\scshape},
}
% NOTA: Evita \textbf{\textsc{...}} porque muchas fuentes no tienen bold small caps.
% Usa \bfseries dentro del contexto \scshape ya aplicado a la tabla.

{\bfseries Carrera} & \SetCell[c=2]{l} \carrera & \\
{\bfseries Asignatura} & \SetCell[c=2]{l} \asignatura & \\
{\bfseries Docente} & \SetCell[c=2]{l} \docenteresponsable & \\
{\bfseries Nivel:} \nivel & {\bfseries Paralelo/Grupo:} \paralelo & {\bfseries Fecha de entrega:} \fechaentrega \\
\SetCell[c=3]{c}{\bfseries Integrantes:} \\
\SetCell[c=3]{l}{\shortstack[l]{\integranteuno \\ \integrantedos \\ \integrantetres \\ \integrantecuatro}} \\
\end{tblr}
}

\begin{center}
  \textbf{Práctica N°: }\numeropractica \\[0.25cm]
  \textsc{\temapractica}
\end{center}

\noindent
\rule{\linewidth}{1pt}